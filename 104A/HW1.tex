
\documentclass[12pt]{article}
\usepackage{amssymb}
\usepackage{amscd}
\usepackage{amsxtra}
\usepackage{enumitem}
\newcommand{\N}{\mathbb{N}}
\newcommand{\C}{\mathbb{C}}
\newcommand{\R}{\mathbb{R}}
\newcommand{\Z}{\mathbb{Z}}
\newcommand{\Q}{\mathbb{Q}}
\newcommand{\ep}{\varepsilon}
\newcommand{\set}[1]{\left\{ #1\right\}}
\newenvironment{proof}{\noindent{\bf Proof.}}{\hfill $\square$\medskip}

\usepackage[utf8]{inputenc}






\title{Math 104A Homework 1}
\author{Rad Mallari}

\begin{document}
\maketitle

\section{Problem 1}
Review and state the following theorems of Calculus:
\begin{enumerate}[label=\textbf{(\alph*)}]
    \item The Intermediate Value Theorem
    \item The Mean Value Theorem
    \item Rolle's Theorem
    \item The Mean Value Theorem for Integrals
    \item The Weighted Mean Value Theorem for Integrals
    \item The Taylor's theorem with Lagrange remainder (Alternative for of Taylor's theorem)
\end{enumerate}

\begin{proof}
\begin{enumerate}[label=\textbf{(\alph*)}]
    \item If $f$ is a continuous real-valued function on an interval $I$, then $f$ has the intermediate value property on $I$: Whenever $a,b\in I$, $a < b$ and $y$ lies between $f(a)$ and $f(b)$ [i.e., $f(a)<y<f(b)$ or $f(b)<y<f(a)$], there exists at least one $x$ in $(a,b)$ such that $f(x)=y$.
    \item Let $f$ be continuous function on $[a,b]$ that is differentiable on $(a,b)$ and satisfies $f(a)=f(b)$. There exists at least one $x$ in $(a,b)$ such that $$f'(x)=\frac{f(b)-f(a)}{b-a}$$
    \item Let $f$ be continuous function on $[a,b]$ that is differentiable on $(a,b)$ and satisfies $f(a)=f(b)$. There exists at least one $x$ in $(a,b)$ such that $f'(x)=0$.
    \item  Let $f$ be continuous function on $[a,b]$ that is differentiable on $(a,b)$ and satisfies $f(a)=f(b)$. There exists at least one $x$ in $(a,b)$ such that $$f(x)=\frac{1}{b-a}\int_{a}^{b}f(t)dt$$
    \item Suppose $f(x), g(x)\in [a,b]$, is integrable, does not change sign, and $f(x)$ is continuous. Then there exists $\eta\in (a,b)$ such that $$\int_{a}^{b}f(x)g(x)=f(\eta)\int_{a}^{b}g(x)dx$$
    \item If $f(x)$ is $n+1$ times continuously differentiable $f\in C^{n+1}$ on an interval containing $a$, then $$f(x)=\sum_{k=0}^{n}\frac{f^{(k)}(a)}{k!}(x-a)^{k}+R_{n+1}(x)$$
    where the remainder is
    $$R_{n+1}(x)=\int_{a}^{x}\int_{a}^{x_{1}}...\int_{a}{x_{n}}f^{n+1}(x_{n+1})dx_{n+1}...dx_{2}dx_{1}$$
\end{enumerate}
\end{proof}

\newpage
\section{Problem 2}
For the function $f(x)=x^{4}-4x^{3}-1$ and the interval $[a,b]=[0,2]$, find the number $\xi$ that occurs in the mean value theorem.

\begin{proof}
Since our function is clearly continuous and $f'$ exists, i.e. $f'(x)=4x^{3}-12x^{2}$, we can use $f(x)-f(c)=f'(\xi)(x-c)$ where in this case, $x=2$ and $c=0$. Therefore, using this formula we have that $f(2)-f(0)=f'(\xi)(2-0)$ which is equal to $f'(\xi)=-9$.
\end{proof}

\section{Problem 3}
Prove that if $f\in \C^{n}(\R)$ and $f(x_{0})=f(x_{1})=...=f(x_{n})$ for $x_{0}<x_{1}<...<x_{n}$, then $f^{(n)}(\xi)=0$ for some $\xi\in(x_{0},x_{n})$.

\begin{proof}
Using induction, we have the proposition that $P(n):$ ``if $f\in \C^{n}(\R)$ and $f(x_{0})=f(x_{1})=...=f(x_{n})$ for $x_{0}<x_{1}<...<x_{n}$, then $f^{(n)}(\xi)=0$". So beginning with the base case where $n=1$, we have that $f(x_{0})=f(x_{1})=0$ where $f$ is continuous. Then by Rolle's theorem, there exists $\xi_{1}\in (0,1)$ where $f'(\xi_{1})=0$. Now assuming our proposition $P(n)$ is true for all $n$, we check that $P(n+1)$ holds. So now our proposition is
\end{proof}

\section{Problem 4}
Derive the Taylor series with remainder term for $\ln(1+x)$ about $1$. Derive an inequality that gives the number of terms that must be taken to yield $\ln4$ with error tems less than $2^{-10}$.

\begin{proof}
Letting $f(x)=\ln(1+x)$, we find the first few derivatives $f$ which are $f'(x)=\frac{1}{1+x}$, $f''(x)=\frac{-1}{(1+x)^{2}}$, $f'''(x)=\frac{2}{(1+x)^{3}}$. For this problem we are given that $a=1$ so $f'(1)=\frac{1}{2}$, $f''(1)=-\frac{1}{2^{2}}$, $f'''(1)=\frac{2}{2^{3}}$. So our Taylor series is $$f(x)=\ln(2)+\frac{1}{2}(x-1)+\frac{-1}{2^{2}*2!}(x-1)^{2}+\frac{2}{2^{3}*3!}(x-1)^{3}+...$$
\end{proof}
\end{document}