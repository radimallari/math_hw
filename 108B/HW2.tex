\documentclass[12pt]{article}
\usepackage{amssymb}
\usepackage{amscd}
\usepackage{amsxtra}
\usepackage{amsmath}
\usepackage{enumitem}
\newcommand{\N}{\mathbb{N}}
\newcommand{\C}{\mathbb{C}}
\newcommand{\R}{\mathbb{R}}
\newcommand{\Z}{\mathbb{Z}}
\newcommand{\Q}{\mathbb{Q}}
\newcommand{\ep}{\varepsilon}
\newcommand{\set}[1]{\left\{ #1\right\}}
\newenvironment{proof}{\noindent{\bf Proof.}}{\hfill $\square$\medskip}

\usepackage[utf8]{inputenc}


\title{Math 108B Homework 3}
\author{Rad Mallari}

\begin{document}
\maketitle
Let $V$ be an inner product space over $F$, and $T\in \mathcal{L}(V)$
\section{Problem 1}
Prove that if $U=\text{range }(T)$, then $U^{\perp}=\text{null }T^{*}$.

\begin{proof}

\end{proof}


\section{Problem 2}
Suppose $P\in\mathcal{L}(V)$ is such that $P^{2}=P$. Prove that $P$ is an orthogonal projection if and only if $P$ is self-adjoint.

\begin{proof}

\end{proof}


\section{Problem 3}
Prove that if $T$ is normal, then $\text{range }T=\text{range }T^{*}$

\begin{proof}
Since $T$ is normal we know that
$$\text{range }T=(\text{null }T^{*})^{\perp}=(\text{null }T)^{\perp}=\text{range }T^{*}$$.
\end{proof}


\section{Problem 4}
Prove that if $T$ is normal, then
$$\text{null }T^{k}=\text{null }T\quad \text{and}\quad \text{range }T^{k}=\text{range }T$$
for every positive integer $k$.

\begin{proof}
If $k=1$, then this is trivial, and now since $k$ is a positive integer, we consider $k\geq 2$. For the first, we know that if $v\in\text{null }T$, then $T^{k}v=T^{k-1}(Tv)=T^{k-1}0=0$.
\end{proof}


\section{Problem 5}
Prove that there does not exist a self-adjoint operator $T\in\mathcal{L}(\R^{3})$ such that $T(1,2,3)=(0,0,0)$ and $T(2,5,7)=(2,5,7)$

\begin{proof}

\end{proof}


\section{Problem 6}
Give a counterexample to show that the product of two self-adjoint operators is not necessarily self-adjoint.

\begin{proof}

\end{proof}


\section{Problem 7}
Suppose $F=\C$. Prove that a normal operator on $V$ is self-adjoint if and only if all its eigenvalues are real.

\begin{proof}

\end{proof}


\section{Problem 8}
Suppose $F=\C$ and $T$ is a normal operator on $V$. Prove that there is a $S\in\mathcal{L}(V)$ such that $T=S^{2}$.

\begin{proof}

\end{proof}


\section{Problem 9}
Prove that if $T$ is a positive operator on $V$, then $T^{k}$ is positive for every positive integer $k$.

\begin{proof}

\end{proof}


\section{Problem 10}
Suppose $T$ is a positive operator on $V$. Prove that $T$ is invertible if and only if $\langle Tx,x\rangle$ is positive for $x\in V\setminus \set{0}$.

\begin{proof}

\end{proof}


\section{Problem 11}
Prove that if $S\in\mathcal{L}(\R^{3})$ is an isometry, then there exists a nonzero vector $x\in\R^{3}$ such that $S^{2}x=x$.

\begin{proof}

\end{proof}

\end{document}