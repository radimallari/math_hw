\documentclass[12pt]{article}
\usepackage{amssymb}
\usepackage{amscd}
\usepackage{amsxtra}
\usepackage{amsmath}
\usepackage{enumitem}
\newcommand{\N}{\mathbb{N}}
\newcommand{\C}{\mathbb{C}}
\newcommand{\R}{\mathbb{R}}
\newcommand{\Z}{\mathbb{Z}}
\newcommand{\Q}{\mathbb{Q}}
\newcommand{\ep}{\varepsilon}
\newcommand{\set}[1]{\left\{ #1\right\}}
\newenvironment{proof}{\noindent{\bf Proof.}}{\hfill $\square$\medskip}

\usepackage[utf8]{inputenc}


\title{Math 108B Homework 4}
\author{Rad Mallari}

\begin{document}
\maketitle
For all problems:
\begin{itemize}
    \setlength{\itemindent}{2em}
    \item $V$ is a vector space over $\C$
    \item $\dim V=n$
    \item $x,y\in V$
    \item $V,T,S,N\in\mathcal{L}(V)$
    \item $N\in\mathcal{L}(V)$ is nilpotent if $N^{k}=0$ for some $k>0$.
\end{itemize}
\section{Problem 1}
Prove that if $T^{m-1}x\neq 0$, $T^{m}x=0$, then the set $\set{x,Tx,...,T^{m-1}x}$ is linearly independent.

\begin{proof}

\end{proof}


\section{Problem 2}
Prove that if $ST$ is nilpotent, then $TS$ is nilpotent.

\begin{proof}

\end{proof}


\section{Problem 3}
Prove that if $N$ is nilpotent, then $0$ is the only eigenvalue of $N$.

\begin{proof}

\end{proof}


\section{Problem 4}
Prove that if $\text{null }N^{n-1}\neq\text{null }N^{n}$, then $N$ is nilpotent.

\begin{proof}

\end{proof}


\section{Problem 5}
Prove that if $\text{null }N^{n-2}\neq\text{null }N^{n-1}$, then $N$ has at most two distinct eigenvalues.

\begin{proof}

\end{proof}


\section{Problem 6}
Prove that for any $T$,
$$\text{null }T^{n}\cap\text{range }T^{n}=\set{0}$$

\begin{proof}

\end{proof}


\section{Problem 7}
Find a $3\times3$ matrix whose minimal polynomial is $z^{2}$.

\begin{proof}

\end{proof}



\section{Problem 8}
Find a $4\times4$ matrix whose minimimal polynomial is $z(z-1)^{2}$.

\begin{proof}

\end{proof}


\section{Problem 9}
Suppose $T$ is invertible. Prove that there is a polynomial $p$ such that $T^{-1}=p(T)$.

\begin{proof}

\end{proof}


\section{Problem 10}
Prove that $V$ has a basis consisting of eigenvectors of $T$ if and only if the minimal polynomial of $T$ has no repeated roots.

\begin{proof}

\end{proof}


\section{Problem 11}
Suppose $x\neq 0$. Let $p$ be the monic polynomial of the smallest degree such that $p(T)x=0$. Prove that $p$ divides the minimal polynomial of $T$.

\begin{proof}

\end{proof}


\end{document}