%%%%%%%%%%%%%%%%%%%%%%%%%%%%%%%%%%%%%%%%%
% Arsclassica Article
% LaTeX Template
% Version 1.1 (1/8/17)
%
% This template has been downloaded from:
% http://www.LaTeXTemplates.com
%
% Original author:
% Lorenzo Pantieri (http://www.lorenzopantieri.net) with extensive modifications by:
% Vel (vel@latextemplates.com)
%
% License:
% CC BY-NC-SA 3.0 (http://creativecommons.org/licenses/by-nc-sa/3.0/)
%
%%%%%%%%%%%%%%%%%%%%%%%%%%%%%%%%%%%%%%%%%

%----------------------------------------------------------------------------------------
%	PACKAGES AND OTHER DOCUMENT CONFIGURATIONS
%----------------------------------------------------------------------------------------

\documentclass[
10pt, % Main document font size
a4paper, % Paper type, use 'letterpaper' for US Letter paper
oneside, % One page layout (no page indentation)
%twoside, % Two page layout (page indentation for binding and different headers)
headinclude,footinclude, % Extra spacing for the header and footer
BCOR5mm, % Binding correction
]{scrartcl}

\input{structure.tex} % Include the structure.tex file which specified the document structure and layout

\hyphenation{Fortran hy-phen-ation} % Specify custom hyphenation points in words with dashes where you would like hyphenation to occur, or alternatively, don't put any dashes in a word to stop hyphenation altogether

%----------------------------------------------------------------------------------------
%	TITLE AND AUTHOR(S)
%----------------------------------------------------------------------------------------

\title{\normalfont\spacedallcaps{Non-Negative Matrix Factorization and Applications}} % The article title

\author{\spacedlowsmallcaps{Piper Morris, Brandon Bonifas-Reyes \& Rad Mallari\textsuperscript{1}}} % The article author(s) - author affiliations need to be specified in the AUTHOR AFFILIATIONS block

%----------------------------------------------------------------------------------------

\begin{document}

%----------------------------------------------------------------------------------------
%	HEADERS
%----------------------------------------------------------------------------------------

\renewcommand{\sectionmark}[1]{\markright{\spacedlowsmallcaps{#1}}} % The header for all pages (oneside) or for even pages (twoside)
\lehead{\mbox{\llap{\small\thepage\kern1em\color{halfgray} \vline}\color{halfgray}\hspace{0.5em}\rightmark\hfil}} % The header style

\pagestyle{scrheadings} % Enable the headers specified in this block

%----------------------------------------------------------------------------------------
%	TABLE OF CONTENTS & LISTS OF FIGURES AND TABLES
%----------------------------------------------------------------------------------------

\maketitle % Print the title/author/date block
\setcounter{tocdepth}{2} % Set the depth of the table of contents to show sections and subsections only
\tableofcontents % Print the table of contents
\listoffigures % Print the list of figures
\listoftables % Print the list of tables

%----------------------------------------------------------------------------------------
%	ABSTRACT
%----------------------------------------------------------------------------------------

\section*{Abstract} % This section will not appear in the table of contents due to the star (\section*)
Non-negative matrix factorization, often referred to as NMF, is one of many unsupervised learning algorithms used to extract significant figures from large sets of data.
It is a relatively new process that has become an important tool in dimensionality reduction, and is comparable to other factorization techniques such as principal component analysis (PCA) and vector quantification (VQ).
What makes NMF unique is its constraint on non-negative elements resulting in an often preferable outcome.
NMF has naturally garnered popularity in industries such as machine learning because the algorithms efficiently extract sparse data and provide respective factors that one can easily interpret.

%----------------------------------------------------------------------------------------
%	AUTHOR AFFILIATIONS
%----------------------------------------------------------------------------------------

\let\thefootnote\relax\footnotetext{\textsuperscript{1} \textit{Department of Mathematics, University of California, Santa Barbara}}

%----------------------------------------------------------------------------------------

\newpage % Start the article content on the second page, remove this if you have a longer abstract that goes onto the second page

%----------------------------------------------------------------------------------------
%	INTRODUCTION
%----------------------------------------------------------------------------------------

\section{Introduction}

A statement requiring citation \cite{Figueredo:2009dg}.

Given an $n\times m$ non-negative matrix $A$, NMF aims to express $A$ as two similarly non-negative matrices of smaller dimensions, $W$ and $H$.
Note that when referring to a non-negative matrix, we are discussing a matrix in which all its entries are real and greater than or equal to zero.
$W$ is an $m\times r$ matrix consisting of $k$ basis elements of $A$ and $H$  is a $r\times n$ matrix consisting of the coefficients, or weights, related to the entries of $W$.
We denote some $r>0$ as the inner dimension of $W$ and $H$ and the smallest such instance where $r$ holds is called the \emph{factorization rank}.
While there are several ways to compute this rank to be discussed later, it is important to note that $r$ must be at least the size of the rank of $A$.
Because our resulting NMF is a nonnegative approximation, $W,H$ (holding the same properties) may not be unique either.
Nonetheless, we utilize methods for measuring the error of our approximations.
 
%----------------------------------------------------------------------------------------
%	COST FUNCTIONS
%----------------------------------------------------------------------------------------

\section{Cost Functions}
As mentioned above, the problem of complexity with NMF arises out of the fact that its algorithms attempt to replicate the original matrix, namely $A$.
Thus, the product of $WH$ will be a representation of $A$ and not an exact replica.
As a result, we implement various methods to go about solving an NMF problem.
One of the most common ways to execute this is by using a cost function of which there are several.
They share a common goal in that they attempt to minimize the error between our original matrix and the product of our factorization.
One useful method is accomplished by minimizing the square of the Euclidean distance.
This is also often referred to as minimizing the Frobenius norm as it relates specifically to matrices.
$$\lVert A-WH\rVert^{2}_{F}=\sum_{ij}(A-WH)_{ij}$$

This measure is often chosen because it is not as expensive as other cost functions in terms of calculation.
It also has a key property in that it remains invariant under rotations/orthogonal transformations.
Second, it accounts for the presence of Gaussian noise.
%----remove this sentence?
(It is also recognized as the Euclidean norm because it compiles all the rows/columns of a matrix and concatenates them to produce one vector.) <- Delete??
%-----
Here, one uses iterative update rules to exchange a randomized $WH$ matrix with one that has a smaller error.
We continue to do this until the error meets a predetermined requirement.

An alternative norm used in the process of decomposition is the Kullback Liebler divergence:

%----needs citation
$$D_{KL}(A\vert WH):=\sum_{i=1}^{m}\sum_{j=1}^{n}((WH)_{ij}-A_{ij}\log(WH)_{ij})+\sum_{i=1}^{m}\sum_{j=1}^{n}(A_{ij}\log A_{ij}-A_{ij})$$
%----needs more explanation


% \begin{enumerate}[noitemsep] % [noitemsep] removes whitespace between the items for a compact look
% \item First item in a list
% \item Second item in a list
% \item Third item in a list
% \end{enumerate}
%%%%%%%%%%%%%%%%%%%%%%%%%%%%%%%%%%%%%%%%%%%%%%%%%%%
%------------------------------------------------

%\subsection{Paragraphs}
%
%\lipsum[6] % Dummy text
%
%\paragraph{Paragraph Description} \lipsum[7] % Dummy text
%
%\paragraph{Different Paragraph Description} \lipsum[8] % Dummy text
%
%------------------------------------------------

%\subsection{Math}
%
%\lipsum[4] % Dummy text
%
%\begin{equation}
%\cos^3 \theta =\frac{1}{4}\cos\theta+\frac{3}{4}\cos 3\theta
%\label{eq:refname2}
%\end{equation}
%
%\lipsum[5] % Dummy text
%
%\begin{definition}[Gauss] 
%To a mathematician it is obvious that
%$\int_{-\infty}^{+\infty}
%e^{-x^2}\,dx=\sqrt{\pi}$. 
%\end{definition} 
%
%\begin{theorem}[Pythagoras]
%The square of the hypotenuse (the side opposite the right angle) is equal to the sum of the squares of the other two sides.
%\end{theorem}

%\begin{proof} 
%We have that $\log(1)^2 = 2\log(1)$.
%But we also have that $\log(-1)^2=\log(1)=0$.
%Then $2\log(-1)=0$, from which the proof.
%\end{proof}
%%%%%%%%%%%%%%%%%%%%%%%%%%%%%%%%%%%%%%%%%%%%%%%%%%%%%%%%%%%%%%
%----------------------------------------------------------------------------------------
%	MULTIPLICATIVE UPDATE RULE
%----------------------------------------------------------------------------------------

\section{Multiplicative Update Rule}

%Reference to Figure~\vref{fig:gallery}. % The \vref command specifies the location of the reference
%
%\begin{figure}[tb]
%\centering 
%\includegraphics[width=0.5\columnwidth]{GalleriaStampe} 
%\caption[An example of a floating figure]{An example of a floating figure (a reproduction from the \emph{Gallery of prints}, M.~Escher,\index{Escher, M.~C.} from \url{http://www.mcescher.com/}).} % The text in the square bracket is the caption for the list of figures while the text in the curly brackets is the figure caption
%\label{fig:gallery} 
%\end{figure}

% cite Lee and Seung paper?
A popular approach to NMF problems is the multiplicative update rule created by Lee and Seung.
The multiplicative update rule became popularized because of its simplicity;
however, it's criticized for its slow convergence and because there is no guarantee of convergence to a stationary point which is necessary to find a local minimum.

% also cite
Formula is given by:
$$W\leftarrow W\cdot\frac{(VH^{T})}{(WHH^{T})}$$
$$H\leftarrow H\cdot\frac{(W^{T}V)}{W^{T}WH}$$
%------------------------------------------------


%----------------------------------------------------------------------------------------
%	APPLICATIONS
%----------------------------------------------------------------------------------------
\section{Applications}

% add images of python code?
\lipsum[2]

\subsection{Image Processing}
NMF often begins its canonical relationship with the real world in the field of image processing. We take a simple image of a face with p pixels, and reduce the dimensions of the data to a single vector such that the ith entry represents the ith pixel. Then we can allow the rows of our matrix A to represent the p pixels and n columns respectively output one image. The process of NMF, as we have defined it, will produce W and H to be multiplied, where the columns of W embody the basis images and H gives instruction on how to sum up the aforementioned images. All this to reconstruct an approximation of the originally provided face.

\subsection{Text Mining}
We utilize NMF in the field of topic recovery/document classification.
Let each column of A correspond to a text and each row relate to a keyword.
Construct every $(i,j)$-th entry in such a way where it represents every time a given word $(i)$ comes up in a document $(j)$.
%needs citation
This is a type of quality of term frequency construction in such a model, namely the bag-of-words model, where each document being analyzed has its respective set of words.
The ordering of words here is not accounted for.
The concept of sparseness appears once more in A due to the nature of texts/documents which only use a portion of the dictionary.
In any case, NMF provides a decomposition of rank r as follows:
%needs citation
$$A(i,j)\approx\sum_{k=1}^{r}W(i,k)\cdot\sum H(k,j)$$
Note that in the above $A$ is our $j$-th document, $W$ our $k$-th topic, and $H$ conveys the relevance of a topic in a given document. 
\subsection{Facial Recognition}
\lipsum[8]



% ------------------------------------------------
% \subsection{Subsection}

% \lipsum[11] % Dummy text

% \subsubsection{Subsubsection}

% \lipsum[12] % Dummy text

% \begin{description}
% \item[Word] Definition
% \item[Concept] Explanation
% \item[Idea] Text
% \end{description}

% \lipsum[12] % Dummy text

% \begin{itemize}[noitemsep] % [noitemsep] removes whitespace between the items for a compact look
% \item First item in a list
% \item Second item in a list
% \item Third item in a list
% \end{itemize}

% \subsubsection{Table}

% \lipsum[13] % Dummy text

% \begin{table}[hbt]
% \caption{Table of Grades}
% \centering
% \begin{tabular}{llr}
% \toprule
% \multicolumn{2}{c}{Name} \\
% \cmidrule(r){1-2}
% First name & Last Name & Grade \\
% \midrule
% John & Doe & $7.5$ \\
% Richard & Miles & $2$ \\
% \bottomrule
% \end{tabular}
% \label{tab:label}
% \end{table}

% Reference to Table~\vref{tab:label}. % The \vref command specifies the location of the reference

% ------------------------------------------------

% \subsection{Figure Composed of Subfigures}

% Reference the figure composed of multiple subfigures as Figure~\vref{fig:esempio}. Reference one of the subfigures as Figure~\vref{fig:ipsum}. % The \vref command specifies the location of the reference

% \lipsum[15-18] % Dummy text

% \begin{figure}[tb]
% \centering
% \subfloat[A city market.]{\includegraphics[width=.45\columnwidth]{Lorem}} \quad
% \subfloat[Forest landscape.]{\includegraphics[width=.45\columnwidth]{Ipsum}\label{fig:ipsum}} \\
% \subfloat[Mountain landscape.]{\includegraphics[width=.45\columnwidth]{Dolor}} \quad
% \subfloat[A tile decoration.]{\includegraphics[width=.45\columnwidth]{Sit}}
% \caption[A number of pictures.]{A number of pictures with no common theme.} % The text in the square bracket is the caption for the list of figures while the text in the curly brackets is the figure caption
% \label{fig:esempio}
% \end{figure}

%----------------------------------------------------------------------------------------
%	BIBLIOGRAPHY
%----------------------------------------------------------------------------------------

\renewcommand{\refname}{\spacedlowsmallcaps{References}} % For modifying the bibliography heading

\bibliographystyle{unsrt}

\bibliography{sample.bib} % The file containing the bibliography

%----------------------------------------------------------------------------------------

\end{document}