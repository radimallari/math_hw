\documentclass[11pt]{article}
\usepackage{amssymb}
\usepackage{amscd}
\usepackage{amsxtra}
\usepackage{amsmath}
\usepackage{enumitem}
\newcommand{\N}{\mathbb{N}}
\newcommand{\C}{\mathbb{C}}
\newcommand{\R}{\mathbb{R}}
\newcommand{\Z}{\mathbb{Z}}
\newcommand{\Q}{\mathbb{Q}}
\newcommand{\ep}{\varepsilon}
\newcommand{\set}[1]{\left\{ #1\right\}}
\newcommand{\norm}[1]{\left\lVert#1\right\rVert}
\newenvironment{proof}{\noindent{\bf Proof.}}{\hfill $\square$\medskip}

\usepackage[utf8]{inputenc}


\title{Math 119A Homework 3}
\author{Rad Mallari}

\begin{document}
\maketitle

\section{Problem 1}
Prove or disprove that $x(t)=0,y(t)=3e^{2t}$ is a solution to the following
initial value problem:
\begin{equation}
    \begin{split}
        x^{\prime}&=-x,\\
        y^{\prime}&=x+2y;\\
        x(0)&=0,y(0)=3
    \end{split}
\end{equation}

\begin{proof}
    To check our solution we first check that the derivatives of $x(t)$ and $y(t)$ satisfy
    our differential equation:
    \begin{equation}
        \begin{split}
            x^{\prime}(t)&=0=x\\
            y^{\prime}(t)&=6e^{2t}=0+2(3e^{2t})=x+2y
        \end{split}
    \end{equation}
    Which shows that $x(t),y(t)$ are solutions to our differential equations.
Now plugging in our initial values for $t=0$, we get
\begin{equation}
    \begin{split}
        x(0)&=0\\
        y(0)&=0+e^{0}=3
    \end{split}
\end{equation}
Which also satisfies our initial values, therefore $x(t)=0,y(t)=3e^{2t}$ is a solution
to our differential equation.
\end{proof}

\section{Problem 2}
Prove or disprove that
$$A=\begin{bmatrix}
        \frac{5}{3} & \frac{1}{3} \\
        2           & 0
    \end{bmatrix}$$
is one solution to $x^{\prime}=Ax$ where $x(t)=(e^{2t}-e^{-t},e^{2t}+2e^{-t})$.

\begin{proof}
First, $x^{\prime}(t)$ is
$$x^{\prime}(t)=\begin{bmatrix}
    2e^{2t}+e^{-t}\\
    2e^{2t}-2e^{-t}
\end{bmatrix}$$
Now $Ax$ is
\begin{equation}
    \begin{split}
        Ax&=\begin{bmatrix}
            \frac{5}{3}&\frac{1}{3}\\
            2&0
        \end{bmatrix}
        \begin{bmatrix}
            e^{2t}-e^{-t}\\
            e^{2t}+2e^{-t}
        \end{bmatrix}\\
        &=\begin{bmatrix}
            \frac{5}{3}\left(e^{2t}-e^{-t}\right)+\frac{1}{3}\left(e^{2t}+2e^{-t}\right)\\
            2\left(e^{2t}-e^{-t}\right)+0\left(e^{2t}+2e^{-t}\right)
        \end{bmatrix}\\
        &=\begin{bmatrix}
            \frac{5}{3}e^{2t}-\frac{5}{3}e^{-t}+\frac{1}{3}e^{2t}+\frac{2}{3}e^{-t}\\
            2e^{2t}-2e^{-t}
        \end{bmatrix}\\
        &=\begin{bmatrix}
            \frac{6}{3}e^{2t}-\frac{3}{3}e^{-t}\\
            2e^{2t}-2e^{-t}
        \end{bmatrix}\\
        &=\begin{bmatrix}
            2e^{2t}-e^{-t}\\
            2e^{2t}-2e^{-t}
        \end{bmatrix}
    \end{split}
\end{equation}
Showing that the book answer is incorrect because the first element of the resulting $Ax$
has the incorrect sign in the second term.
\end{proof}

\section{Problem 3}
Show that all eigenvalues are positive is the condition on eigenvalues that is equivalent to
$\lim_{t\to\infty}\left|x(t)\right|=\infty$ for every solution $x(t)$ to $x^{\prime}=Ax$.

\begin{proof}

\end{proof}

\section{Problem 4}
Show that $b>0$ is an assumption required to ensure that $\lim_{t\to\infty}x(t)=0$
for every solution $x(t)$ if $b^{2}-4c>0$.

\begin{proof}

\end{proof}

\section{Problem 5}
Prove or disprove that $x(t)=3e^{t}\cos2t+9e^{t}\sin2t,y(t)=3e^{t}\sin2t-9e^{t}\cos2t$
is a solution to the following initial value problem:
\begin{equation}
    \begin{split}
        x^{\prime}&=Ax,\\
        x(0)&=(3,9);\\
        A&=\begin{bmatrix}
            1 & -2 \\
            2 & 1
        \end{bmatrix}
    \end{split}
\end{equation}

\begin{proof}

\end{proof}

\section{Problem 6}
Prove or disprove that $\dim E=\dim E_{c}$ and $\dim F\geq \dim F_{R}$ are relations
that exist between $\dim E$ and $\dim E_{c}$ and $\dim F$ and $\dim F_{R}$ given that
$E\subset \R^{n}$ and $F\subset C^{n}$ are subspaces.

\begin{proof}

\end{proof}

\section{Problem 7}
Prove or disprove that $\dim\supset R_{CR}$ is a relation between $F$ and $F_{RC}$
given that $F\subset C^{n}$ is any subspace.

\begin{proof}

\end{proof}

\section{Problem 8}
Solve the following initial value problem
\begin{enumerate}[label=(\alph*)]
    \item \begin{equation}
              \begin{split}
                  x^{\prime}&=-y,\\
                  y^{\prime}&=x;\\
                  x(0)&=1,y(0)=1
              \end{split}
          \end{equation}
\end{enumerate}

\begin{proof}
    \begin{enumerate}[label=(\alph*)]
        \item
    \end{enumerate}
\end{proof}

\section{Problem 9}
Solve the initial value problem

\begin{equation}
    \begin{split}
        x^{\prime}&=-4y\\
        y^{\prime}&=x;\\
        x(0)=0,&\quad y(0)=-7
    \end{split}
\end{equation}

\begin{proof}

\end{proof}

\section{Problem 10}
Let $F\subset C^{2}$ be the subspace spanned by the vector $(1,i)$
\begin{enumerate}[label=(\alph*)]
    \item Prove that $F$ is not invariant under conjugation and hence is not
          the complexification of any subspace of $\R^{2}$
    \item Find $F_{\R}$ and $(F_{\R})_{C}$.
\end{enumerate}

\begin{proof}
    \begin{enumerate}[label=(\alph*)]
        \item
        \item
    \end{enumerate}
\end{proof}

\section{Problem 11}
Let $E$ be a real vector space and $T\in L(E)$. Show that $(\ker T)_{C} =\ker(T_{C})$,
$(\text{Im } T)_{C}=\text{Im }(T_{C})$, and $(T^{-1})_{C}=(T_{C})^{-1}$ if $T$ is invertible.

\begin{proof}

\end{proof}
\end{document}