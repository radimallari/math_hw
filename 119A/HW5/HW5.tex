\documentclass[11pt]{article}
\usepackage{amssymb}
\usepackage{amscd}
\usepackage{amsxtra}
\usepackage{amsmath}
\usepackage{enumitem}
\newcommand{\N}{\mathbb{N}}
\newcommand{\C}{\mathbb{C}}
\newcommand{\R}{\mathbb{R}}
\newcommand{\Z}{\mathbb{Z}}
\newcommand{\Q}{\mathbb{Q}}
\newcommand{\ep}{\varepsilon}
\newcommand{\set}[1]{\left\{ #1\right\}}
\newcommand{\norm}[1]{\left\lVert#1\right\rVert}
\newenvironment{proof}{\noindent{\bf Proof.}}{\hfill $\square$\medskip}

\usepackage[utf8]{inputenc}


\title{Math 119A Homework 5}
\author{Rad Mallari}

\begin{document}
\maketitle
\section{Problem 1}
For the following operator $T$ find bases for the general eigenspaces; give the atrices (for the standard basis) of the semisimple and nilpotent parts of $T$.
$$\begin{bmatrix}
  1&1\\0&1  
\end{bmatrix}$$

\begin{proof}
    
\end{proof}

\section{Problem 2}
Identify $R^{n+1}$ with the set $P_{n}$ of polynomials of degree $\leq n$, via the correspondence
$$(a_{n},...,a_{0})\leftrightarrow a_{n}t^{n}+...+a_{1}t+a_{0}$$
Let $D:P_{n}\to P_{n}$ be the differentiation operator. Prove $D$ is nilpotent.

\begin{proof}
  
\end{proof}

\section{Problem 3}
Find the matrix of $D$ in the standard basis in \textbf{Problem 2}

\begin{proof}
  
\end{proof}

\section{Problem 4}
Classify the following operators on $\R^{4}$ by similarity
\begin{enumerate}[label=(\alph*)]
  \item $\begin{bmatrix}
    0&1&0&0\\
    0&0&2&0\\
    0&0&0&3\\
    0&0&0&0
  \end{bmatrix}$
  \item $\begin{bmatrix}
    2&0&0&2\\
    0&0&0&0\\
    0&0&0&0\\
    -2&0&0&-2
  \end{bmatrix}$
  \item $\begin{bmatrix}
    0&0&0&0\\
    4&0&0&0\\
    0&0&0&4\\
    0&0&0&0
  \end{bmatrix}$
  \item $\begin{bmatrix}
    0&0&0&0\\
    1&0&0&0\\
    0&1&0&0\\
    -1&-1&-1&0
  \end{bmatrix}$
  \item $\begin{bmatrix}
    0&0&0&100\\
    0&0&0&0\\
    0&0&0&0\\
    0&0&0&0
  \end{bmatrix}$
\end{enumerate}

\begin{proof}
  
\end{proof}

\section{Problem 5}
Let $A$ be a $3\times 3$ real matrix which is not diagonal. If $(A+I)^{2}=O$, find the real canonical form of $A$.

\begin{proof}
  
\end{proof}

\section{Problem 6}
Every $n\times n$ matrix is similar to its transpose.

\begin{proof}
  
\end{proof}

\section{Problem 7}
Let $A\in L(\R^{2})$. Suppose all solutions of $x^{\prime}=Ax$ are periodic with the same period. Then $A$ is semisimple and the characteristic polynomial is a power of $l^{2}+a^{2}$, $a\in\R$.

\begin{proof}
  
\end{proof}

\section{Problem 8}
Find a map $s:\R\to\R$ such that
$$s^{(3)}-s^{(2)}+4s^{\prime}-4s=0$$
$$(s0)=1,\quad s^{\prime}(0)=-1,\quad s^{\prime\prime}(0)=1$$

\begin{proof}
  
\end{proof}

\section{Problem 9}
Consider the equation
$$s^{(4)}+4s^{(3)}+5s^{(2)}+4s^{\prime}+4s=0$$
Find out for which initial conditions $s(0)$, $s^{\prime}(0)$, $s^{\prime\prime}(0)$ there is a solution $s(t)$ such that $s(t)$ is periodic

\begin{proof}
  
\end{proof}

\section{Problem 10}
If $e^{tB}$ and $e^{tA}$ are both contractions on $\R^{n}$, and $BA=AB$, then $e^{t(A+B)}$ is a contraction. Similarly for expansions.

\begin{proof}
  
\end{proof}

\section{Problem 11}
Show that for \textbf{Problem 10} can be false if the assumption that $AB=BA$ is dropped.

\begin{proof}
  
\end{proof}

\section{Problem 12}
$e^{tA}$ is hyperbolic if and only if for each $x\neq0$ either
$$|e^{tA}x|\to\infty\quad\text{as}\quad t\to\infty$$
or
$$|e^{tA}x|\to\infty\quad\text{as}\quad t\to-\infty$$

\begin{proof}
  
\end{proof}

\section{Problem 13}
Show that a hyperbolic flow has no nontrivial periodic solutions.

\begin{proof}
  
\end{proof}

\section{Problem 14}
For each of the following properties defines a set of real $n\times n$ matrices. Find out which sets are dense, and which are open in the space $L(\R^{n})$ of all linear operators on $\R^{n}$:
\begin{enumerate}[label=(\alph*)]
  \item determinant $\neq0$;
  \item trace is rational;
  \item entries are not integers;
  \item $e\leq \text{determinant} <4$
  \item $-1<|\lambda|<1$ for every eigenvalue $\lambda$;
  \item no real eigenvalues;
  \item each real eigenvalue has multiplicity one.
\end{enumerate}

\begin{proof}
  \begin{enumerate}[label=(\alph*)]
    \item 
  \end{enumerate}
\end{proof}

\section{Problem 15}
Which of the following properties of operators on $\R^{n}$ are generic?
\begin{enumerate}[label=(\alph*)]
  \item $|\lambda|\neq1$ for every eigenvalue $\lambda$;
  \item $n=2$; for some eigenvalue is not real;
  \item $n=3$; some eigenvalue is not real;
  \item no solution of $x^{\prime}=Ax$ is periodic (except the zero solution);
  \item there are $n$ distinct eigenvalues with distinct imaginary parts;
  \item $Ax\neq x$ and $Ax\neq-x$ for all $x\neq 0$.
\end{enumerate}

\begin{proof}
  \begin{enumerate}[label=(\alph*)]
    \item 
  \end{enumerate}
\end{proof}
\end{document}

