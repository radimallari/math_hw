\documentclass[12pt]{article}
\usepackage{amssymb}
\usepackage{amscd}
\usepackage{amsxtra}
\usepackage{amsmath}
\usepackage{enumitem}
\newcommand{\N}{\mathbb{N}}
\newcommand{\C}{\mathbb{C}}
\newcommand{\R}{\mathbb{R}}
\newcommand{\Z}{\mathbb{Z}}
\newcommand{\Q}{\mathbb{Q}}
\newcommand{\ep}{\varepsilon}
\newcommand{\set}[1]{\left\{ #1\right\}}
\newenvironment{proof}{\noindent{\bf Proof.}}{\hfill $\square$\medskip}

\usepackage[utf8]{inputenc}


\title{Math 122A Homework 6}
\author{Rad Mallari}

\begin{document}
\maketitle

\section{Problem 1}
Let $C$ be a closed, positive, and simple curve. Using Green's Theorem prove that
$$\frac{1}{2i}\int_{C}\bar{z}dz=\text{area enclosed by }C$$

\begin{proof}
    Since $\bar{z}=x-iy$, this is equivalent to
    $$\frac{1}{2i}\int_{C}(x-iy)(dx+idy)=\frac{1}{2i}\left [\int_{C}(xdx+ydy)+i\int_{C}(xdy-ydx)\right ]$$
    Where $D$ is the area boundead by $C$. Now using Green's Theorem, twice we have that
    $$\frac{1}{2i}\left[\iint_{D}(0-0)dxdy+i\iint_{D}(1-(-1))dxdy\right]$$
    $$=\iint_{D}dxdy$$
    Which after evaluating the integrals is exactly the area enclosed by $C$.
\end{proof}


\section{Problem 2}
Consider the function $f(z)=(z+1)^{2}$ and the region $R$ bounded by the triangle with vertices $0,2,i$ (its boundary and interior). Find the points where $\left |f(z)\right |$ reaches its maximum and minimum value of $R$.

\begin{proof}

\end{proof}


\section{Problem 3}
Find the maximum of $\left |\sin(z)\right |$ on $[0,2\pi]\times[0,2\pi]$.

\begin{proof}

\end{proof}


\section{Problem 4}
Calculate:
\begin{enumerate}[label=\textbf{(\alph*)}]
    \item $$\int_{0}^{2\pi}\frac{d\theta}{a+b\cos(\theta)},\quad 0<b<a$$
          HINT: Work backwards using $\cos(\theta)=\frac{e^{i\theta}+e^{-i\theta}}{2}$ to convert the integral into a complex integral along the curve $\left |z\right |=1$
    \item $$\int_{0}^{2\pi}\frac{d\theta}{(a+b\cos(\theta))^{2}}$$
    \item $$\int_{0}^{2\pi}\frac{\sin(\theta)d\theta}{(a+b\cos(\theta))^{2}},\quad 0<b<a$$
\end{enumerate}

\begin{proof}
    \begin{enumerate}[label=\textbf{(\alph*)}]
        \item
        \item
        \item
    \end{enumerate}
\end{proof}


\section{Problem 5}
Prove that if $f:\C\rightarrow\C$ is entire such that for some $n\in\N$
$$\lim_{\left |z\right |\to\infty}\frac{\left |f(z)\right |}{\left |z\right |^{n}}=M<\infty,$$
then $f$ is a polynomial of degree at most $n$.

\begin{proof}

\end{proof}


\section{Problem 6}
Let $A\subset \C$ be an open set and $f:A\to\C$ be an analytic function on $A$. Assuming that $z_{0}\in A$ such that
$$\set{z\in\C:\left |z-z_{0}\right |\leq R},\quad R>0$$
prove that
$$f(z_{0})=\frac{1}{\pi R^{2}}\iint_{\left |z-z_{0}\right|\leq R}f(x+iy)dxdy$$

\begin{proof}

\end{proof}


\section{Problem 7}
Let $f:R\to R$ be defined as
$$f(x)=e^{\frac{-1}{x^{2}}}\quad\text{if}\quad x\neq 0,\quad f(0)=0$$
Show that $f$ is infinitely differentiable $\forall n\in\N$, $f^{(n)}(0)=0$. Verify that the power series of $f$ at $x=0$ does not agree with $f$ in any neighborhood of $0$.

\begin{proof}

\end{proof}

\end{document}