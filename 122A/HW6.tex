\documentclass[12pt]{article}
\usepackage{amssymb}
\usepackage{amscd}
\usepackage{amsxtra}
\usepackage{amsmath}
\usepackage{enumitem}
\newcommand{\N}{\mathbb{N}}
\newcommand{\C}{\mathbb{C}}
\newcommand{\R}{\mathbb{R}}
\newcommand{\Z}{\mathbb{Z}}
\newcommand{\Q}{\mathbb{Q}}
\newcommand{\ep}{\varepsilon}
\newcommand{\set}[1]{\left\{ #1\right\}}
\newenvironment{proof}{\noindent{\bf Proof.}}{\hfill $\square$\medskip}

\usepackage[utf8]{inputenc}


\title{Math 122A Homework 6}
\author{Rad Mallari}

\begin{document}
\maketitle

\section{Problem 1}
Let $C$ be a closed, positive, and simple curve. Using Green's Theorem prove that
$$\frac{1}{2i}\int_{C}\bar{z}dz=\text{area enclosed by }C$$

\begin{proof}
    Since $\bar{z}=x-iy$, this is equivalent to
    $$\frac{1}{2i}\int_{C}(x-iy)(dx+idy)=\frac{1}{2i}\left
    [\int_{C}(xdx+ydy)+i\int_{C}(xdy-ydx)\right ]$$ Where $D$ is the area
    boundead by $C$. Now using Green's Theorem, twice we have that
    $$\frac{1}{2i}\left[\iint_{D}(0-0)dxdy+i\iint_{D}(1-(-1))dxdy\right]$$
    $$=\iint_{D}dxdy$$ Which after evaluating the integrals is exactly the area
    enclosed by $C$.
\end{proof}


\section{Problem 2}
Consider the function $f(z)=(z+1)^{2}$ and the region $R$ bounded by the
triangle with vertices $0,2,i$ (its boundary and interior). Find the points
where $\left |f(z)\right |$ reaches its maximum and minimum value of $R$.

\begin{proof}
    By the Maximum Modulus Theorem, we know that
    $\left|f(z)\right|=\left|z+1\right|^2$. Then the minimum would be at the
    boundary where $\left|f(z)\right|$ is the closest to $-1$ and maximum is the
    furthest. Then we know that since $\left|f(z)\right|=1$ at $z=0$, then this
    is the minimum, and since $\left|f(z)\right|=9$ at $z=2$, then this is the
    maximum.
\end{proof}


\section{Problem 3}
Find the maximum of $\left |\sin(z)\right |$ on $[0,2\pi]\times[0,2\pi]$.

\begin{proof}
    We can rewrite $\sin(z)$ as:
    \begin{equation}
        \begin{split}
            \sin(z)&=\frac{e^{iz}-e^{-iz}}{2i}=\frac{e^{x+iy}-e^{-i(x+iy)}}{2i}\\
            &=\frac{e^{ix}e^{-y}-e^{-ix}e^{y}}{2i}\\
            &=\frac{(\cos x+i\sin x)e^{y}-(\cos x-i\sin x)e^{y}}{2i}\\
            &=-\frac{1}{i}\cos (x)\left(\frac{e^{-y}-e^{y}}{2}\right)+\sin (x)\left(\frac{e^{y}+e^{-y}}{2}\right)\\
            &=i\cos(x)\sinh(y)+\sin(x)\cosh(y)
        \end{split}
    \end{equation}
    And using Maximum Modulus Theorem, and \textbf{(1)}, we have that:
    $$\left|\sin(z)\right|^{2}=\cos^{2}(x)\sinh^{2}(y)+\sin^{2}(x)\cosh^{2}(y)$$
    Using the identities that states $\cosh^{2}(y)-\sinh^{2}(y)=1$ and
    $\cos^{2}(x)+\sin^{2}(x)=1$, we can rewrite this as:
    $$\left|\sin(z)\right|^{2}=\cos^{x}\sinh^{2}(y)+\sin^{2}(x)\cdot(1+\sinh^{2}(y))$$
    Leaving us with
    $$\left|\sin(z)\right|^{2}=\sinh^{2}(y)+\sin^{2}(x)$$ We know that that
    maximum of $\sin^{2}(x)=1$ which is at $x=\frac{\pi}{2}$ and
    $x=\frac{3\pi}{2}$, meanwhile the maximum of $\sinh^{2}(y)$ is located at
    $y=2\pi$. Therefore, out maximum is at the boundaries $2\pi$.
\end{proof}


\section{Problem 4}
Calculate:
\begin{enumerate}[label=\textbf{(\alph*)}]
    \item $$\int_{0}^{2\pi}\frac{d\theta}{a+b\cos(\theta)},\quad 0<b<a$$
          HINT: Work backwards using $\cos(\theta)=\frac{e^{i\theta}+e^{-i\theta}}{2}$ to convert the integral into a complex integral along the curve $\left |z\right |=1$
    \item $$\int_{0}^{2\pi}\frac{d\theta}{(a+b\cos(\theta))^{2}}$$
    \item $$\int_{0}^{2\pi}\frac{\sin(\theta)d\theta}{(a+b\cos(\theta))^{2}},\quad 0<b<a$$
\end{enumerate}

\begin{proof}
    \begin{enumerate}[label=\textbf{(\alph*)}]
        \item By the hint, we work backwards and get 
        $$\int_{0}^{2\pi}\frac{d\theta}{a+b\left(\frac{e^{i\theta}+e^{-i\theta}}{2}\right)}$$
        Factoring out a $2$ gives us
        $$2\int_{0}^{2\pi}\frac{d\theta}{2a+b\left(e^{i\theta}+e^{-i\theta}\right)}$$
        Letting $z(\theta)=e^{i\theta}$, where $z:[0,2\pi]\to\C$, implies
        $d\theta=\frac{dz}{iz}$ and our line integral becomes
        $$I=\frac{2}{i}\oint_{\left|z\right|=1}\frac{dz}{bz^{2}+2az+b}$$ Here
        our $f(z)=\frac{1}{bz^{2}+2az+b}$ is analytic except at
        $\frac{a\pm\sqrt{a^{2}-b^{2}}}{b}$. These two points are:
        $$z_{1}=\frac{a+a\sqrt{1-\frac{b^{2}}{a^{2}}}}{b}\quad\text{and}\quad
        z_{2}=\frac{a-a\sqrt{1-\frac{b^{2}}{a^{2}}}}{b}$$ By our condition that
        $0<b<a$, we know $z_{2}$ must be outside our $z$, therefore letting
        $h(z)=\frac{1}{z-\left(\frac{a-a\sqrt{1-\frac{b^{2}}{a^{2}}}}{b}\right)}$
        and by Cauchy Theorem we get
        $$I=\frac{2}{i}\oint_{\left|z\right|=1}\frac{h(z_{2})dz}{z-\left(\frac{a+a\sqrt{1-\frac{b^{2}}{a^{2}}}}{b}\right)}=\frac{2\pi
        b}{\sqrt{a^{2}-b^{2}}}$$
        \item Similarly, letting $\cos(\theta)=\frac{e^{i\theta}+e^{-\theta}}{2}$ yields
        $$\int_{0}^{2\pi}\frac{d\theta}{\left(a+b\left(\frac{e^{i\theta}+e^{-\theta}}{2}\right)\right)^{2}}$$
        Then letting $z(\theta)=e^{i\theta}$, $d\theta=\frac{dz}{iz}$, we get
        $$\frac{4}{i}\oint_{\left|z\right|=1}\frac{zdz}{(bz^{2}+2az+b)^{2}}$$
        Our singularities here are the same as \textbf{(a)} which are
        $$z_{1}=\frac{a+a\sqrt{1-\frac{b^{2}}{a^{2}}}}{b}\quad\text{and}\quad
        z_{2}=\frac{a-a\sqrt{1-\frac{b^{2}}{a^{2}}}}{b}$$
        But $z_{1}$ in this case is outside of $\left|z\right|=1$.
        \item
    \end{enumerate}
\end{proof}


\section{Problem 5}
Prove that if $f:\C\rightarrow\C$ is entire such that for some $n\in\N$
$$\lim_{\left |z\right |\to\infty}\frac{\left |f(z)\right |}{\left |z\right |^{n}}=M<\infty,$$
then $f$ is a polynomial of degree at most $n$.

\begin{proof}
Since $f$ is analytic, by a theorem in \textbf{Lecture 16} which states that $f$ has a
power series expansion in the neighborhood of analyticity that is
$$f(z_{0})=\sum_{n=0}^{\infty}(z_{0}-z_{1})^{n}\left(\frac{f^{n}(z_{1})}{n!}\right)$$
In our case, $z_{1}=0$ so this becomes
$$f(z_{0})=\sum_{n=0}^{\infty}(z_{0})^{n}\left(\frac{f^{n}(0)}{n!}\right)$$\
By \textbf{Section 49} of the book, we get
$$\left|f^{n}(0)\right|\leq \frac{n! M_{R}}{R^{n}}$$
Where $M_{R}$ denotes the maximum value of $\left|f(z)\right|$.
\end{proof}


\section{Problem 6}
Let $A\subset \C$ be an open set and $f:A\to\C$ be an analytic function on $A$. Assuming that $z_{0}\in A$ such that
$$\set{z\in\C:\left |z-z_{0}\right |\leq R},\quad R>0$$
prove that
$$f(z_{0})=\frac{1}{\pi R^{2}}\iint_{\left |z-z_{0}\right|\leq R}f(x+iy)dxdy$$

\begin{proof}

\end{proof}


\section{Problem 7}
Let $f:R\to R$ be defined as
$$f(x)=e^{\frac{-1}{x^{2}}}\quad\text{if}\quad x\neq 0,\quad f(0)=0$$
Show that $f$ is infinitely differentiable $\forall n\in\N$, $f^{(n)}(0)=0$. Verify that the power series of $f$ at $x=0$ does not agree with $f$ in any neighborhood of $0$.

\begin{proof}

\end{proof}

\end{document}