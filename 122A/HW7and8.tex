\documentclass[11pt]{article}
\usepackage{amssymb}
\usepackage{amscd}
\usepackage{amsxtra}
\usepackage{amsmath}
\usepackage{enumitem}
\newcommand{\N}{\mathbb{N}}
\newcommand{\C}{\mathbb{C}}
\newcommand{\R}{\mathbb{R}}
\newcommand{\Z}{\mathbb{Z}}
\newcommand{\Q}{\mathbb{Q}}
\newcommand{\ep}{\varepsilon}
\newcommand{\set}[1]{\left\{ #1\right\}}
\newenvironment{proof}{\noindent{\bf Proof.}}{\hfill $\square$\medskip}

\usepackage[utf8]{inputenc}


\title{Math 122A Homework 7 and 8}
\author{Rad Mallari}

\begin{document}
\maketitle


\section{Problem 1}
Let $D_{1}(z_{0})=\set{z\in\C:|z-z_{0}|<1}$. Let $f,g:D_{1}(z_{0})\to\C$ be two analytic functions on $D_{1}(z_{0})$. Prove that if
$$f^{(n)}(z_{0})=g^{(n)}(z_{0}),\quad n=0,1,2,3,...$$
then $f(z)=g(z)$, $\forall z\in D_{1}(z_{0})$.

\begin{proof}
    By our given, we know there is a Taylor Series expansion of $f(z)$ and $g(z)$ centered around $z_{0}$ such that
    $$f(z)=\sum_{n=0}^{\infty}(z-z_{0})^{n}\frac{f^{(n)}(z_{0})}{n!} \quad\text{and}\quad g(z)=\sum_{n=0}^{\infty}(z-z_{0})^{n}\frac{g^{(n)}(z_{0})}{n!}$$
    where $n=0,1,2,3,...$ therefore, equating the two we have that
    $$\sum_{n=0}^{\infty}(z-z_{0})^{n}\frac{f^{(n)}(z_{0})}{n!}=\sum_{n=0}^{\infty}(z-z_{0})^{n}\frac{g^{(n)}(z_{0})}{n!}$$
    This reduces to
    $$f^{(n)}(z_{0})=g^{(n)}(z_{0})$$
    Which is exactly what we want.
\end{proof}


\newpage
\section{Problem 2}
Let $D_{1}(z_{0})=\set{z\in\C:|z-z_{0}|<1}$. Let $f:D_{1}(z_{0})\to\C$ be an analytic function on $D_{1}(z_{0})$ \underline{such that is has a zero of $N\in\N$ at $z_{0}$}, i.e.
$$f(z_{0})=f'(z_{0})=...=f^{N-1}(z_{0})=0,\quad f^{n}(z_{0})\neq0$$
\begin{enumerate}[label=\textbf{(\roman*)}]
    \item Prove that there exists $g:D_{1}(z_{0})\to\C$ analytic on $D_{1}(z_{0})$ with $g(z_{0})\neq0$ and
          $$f(z)=(z-z_{0})^{N}g(z)$$
    \item There exists $\delta>0$ such that if $0<|z-z_{0}|<\delta$ such that $f(z)\neq0$. (The zeros of a non-trivial analytic function are isolated)
\end{enumerate}

\begin{proof}
    \begin{enumerate}[label=\textbf{(\roman*)}]
        \item Since we are given that
              $$f(z_{0})=f'(z_{0})=...=f^{N-1}(z_{0})=0,\quad f^{n}(z_{0})\neq0$$
              and letting $z_{0} =0$, we know that we can Taylor expand $f(z)$ such that:
              $$f(z)=\underbrace{\sum_{n=0}^{N-1}\frac{f^{(n)}(z_{0})}{n!}(z-z_{0})^{n}}_\text{=0 (by definition)}+\sum_{k=N}^{\infty}\frac{f^{(k)}(z_{0})}{k!}(z-z_{0})^{k}$$
              where the remaining nonzero sum terms consists of analytic functions. Factoring out a $(z-z_{0})^{N}$ yields:
              $$f(z)=(z-z_{0})^{N}\cdot\sum_{k=0}^{\infty}\frac{f^{(k)}(z_{0})}{k!}(z-z_{0})^{k}$$
              Finally, letting $g(z)=\sum_{k=0}^{\infty}\frac{f^{(k)}(z_{0})}{k!}(z-z_{0})^{k}$ we conclude:
              $$f(z)=(z-z_{0})^{N}\cdot g(z_{0})$$
        \item Taking $f(z)$ in \textbf{Problem 2(i)}, we know that after the first zero terms of the Taylor expansion, we have
              $$f(z)=(z-z_{0})^{N}\cdot g(z_{0})$$
              Clearly, the first term of $g(0)\neq 0$ and is a constant and the following terms are nonzero by definition. So, it follows that there must exist a nonzero $\delta>0$ such that $\left|z-z_{0}\right|<\delta$ which implies that $\left|g(z)\right|\neq0$. Clearly, $(z-z_{0})^{N}\neq0$ so the zeros of a non-trivial analytic function are isolated.
    \end{enumerate}
\end{proof}


\newpage
\section{Problem 3}
Let $f(z)=\sin(\frac{\pi}{z})$. Thus $f(\frac{1}{n})=0$. Does this contradict the result in \textbf{Problem 2}?

\begin{proof}
    We notice that for all possible of $\frac{1}{n}$, $n\in\N$, we have $f(\frac{1}{n})=\sin(n\pi)$ which is $0$ for all $n$. Furthermore as the limit approaches infinity, $\frac{1}{n}$ approaches $0$, and therefore $f(\frac{1}{n})=\sin(0)=0\Rightarrow f(\frac{1}{n})=f'(\frac{1}{n})=...=f^{n}(\frac{1}{n})=0$. This fails our assumption that
    $$f(z_{0})=f'(z_{0})=...=f^{N-1}(z_{0})=0,\quad f^{n}(z_{0})\neq0$$
    and so this does not contradict the result of \textbf{Problem 2}.
\end{proof}


\newpage
\section{Problem 4}
Find the order of each of the zeros of the given functions:
\begin{enumerate}[label=\textbf{(\alph*)}]
    \item $(z^{2}-4z+4)^{2}$
    \item $z^{2}(1-\cos(z))$
    \item $e^{2z}-3e^{z}-4$
\end{enumerate}

\begin{proof}
    Functions $f$ that are analytic at a point $z_{0}$ has a zero of order $m$ at $z_{0}$ if and only if there is a function $g$, which is analytic and nonzero at $z_{0}$ such that
    $$f(z)=(z-z_{0})^{m}g(z)$$
    \begin{enumerate}[label=\textbf{(\alph*)}]
        \item Therefore, we can factor simplify this to get
              $$((z-2)^{2})^{2}=(z-2)^{4}$$
              which makes it clear that we have a $g(z)=0$ and $z_{0}=2$, from which we can conlude we have a zero $m=4$.
        \item Using the Taylor exapnsion of $\cos z$ about $z_{0}=0$, we have that:
              \begin{equation}
                  \begin{split}
                      z^{2}(1-\cos(z))&=z^{2}\left[1-\left(1-\frac{z^{2}}{2!}+\frac{z^{4}}{4!}-\frac{z^{6}}{6!}+...\right)\right]\\
                      &=z^{2}\left(\frac{z^{2}}{2!}-\frac{z^{4}}{4!}+\frac{z^{6}}{6!}+...\right)\\
                      &=z^{4}\left(\frac{1}{2!}-\frac{z^{2}}{4!}+\frac{z^{4}}{6!}+...\right)\quad\text{(factoring out a }z^{2}\text{)}
                  \end{split}
              \end{equation}
              From here, we have the form we wanted where we let our multiplicand be $(z-z_{0})=(z-0)^{4}$, and letting $g(z)$ be the multiplier which is $\frac{1}{2!}$ when $z_{0}=0$, i.e. nonzero. Therefore, our $m$ or the order of zero is $4$.
        \item Similar to \textbf{(a)}, we can factor this to get $(e^{z}-4)(e^{z}-1)$. Now, similar to \textbf{(b)}, we Taylor expand our $e^{z}$ to get:
    \end{enumerate}
\end{proof}


\newpage
\section{Problem 5}
Locate the isolated singularity of the given function and tell whether it is a removable singularity, a pole, or an essential singularity.
\begin{enumerate}[label=\textbf{(\alph*)}]
    \item
          \begin{flushleft}
              $\begin{aligned}
                      \frac{e^{z}-1}{z}
                  \end{aligned}$
          \end{flushleft}
    \item
          \begin{flushleft}
              $\begin{aligned}
                      \frac{z^{2}}{\sin(z)}
                  \end{aligned}$
          \end{flushleft}
    \item
          \begin{flushleft}
              $\begin{aligned}
                      \frac{e^{z}-1}{e^{2z}-1}
                  \end{aligned}$
          \end{flushleft}
    \item
          \begin{flushleft}
              $\begin{aligned}
                      \frac{1}{1-cos(z)}
                  \end{aligned}$
          \end{flushleft}
\end{enumerate}

\begin{proof}
    If a function $f$ has an isolated singular point at $z_{0}$, then it's Laurent series form is:
    $$f(z)=\sum_{n=0}^{\infty}a_{n}(z-z_{0})+\frac{b_{1}}{z-z_{0}}+\frac{b_{2}}{(z-z_{0})^{2}}+....+\frac{b_{n}}{(z-z_{0})^{n}}+...$$
    When all $b_{n}=0$, then we have a removable singular point $z_{0}$. If we have $n\geq1$, where the $b_{n}$ terms are nonzero, and $n$ is finite, then we have a pole of order $n$. Finally if we have an infinite number of $b_{n}$, which are nonzero, then $z_{0}$ is an essential singular point of $f$.
    \begin{enumerate}[label=\textbf{(\alph*)}]
        \item This has a singularity at $z_{0}=0$, therefore taking the Taylor expansion of $e^{z}$ about $z_{0}$ gives:
              \begin{equation}
                  \begin{split}
                      \frac{e^{z}-1}{z}&=\frac{\left(1+z+\frac{z^{2}}{2!}+\frac{z^{3}}{3!}+...\right)-1}{z}\quad\text{(subtracting $1$)}
                      \\
                      &=\frac{z+\frac{z^{2}}{2!}+\frac{z^{3}}{3!}+...}{z}\quad\text{(dividing by $z$)}
                      \\
                      &=1+\frac{z}{2!}+\frac{z^{2}}{3!}+\frac{z^{3}}{4}+...
                  \end{split}
              \end{equation}
              Here it's clear that we do not have $b$ terms since we do not have terms where $(z-z_{0})$ is the denominator. Therefore, $z_{0}=0$ is a removable singular point.
        \item We know that $\sin(z)=0$ for $z_{0}=\pi k$ where $k\in\Z$. Therefore, expanding about $z_{0}$, we get
              \begin{equation}
                  \begin{split}
                      \frac{z^{2}}{\sin(z)}&=\frac{z^{2}}{z-\frac{z^{3}}{3!}+\frac{z^{5}}{5!}-...}\\
                      &=\frac{z^{2}}{z\left(1-\frac{z^{2}}{3!}+\frac{z^{4}}{5!}-...\right)}\quad\text{(factoring a $z$ in the denominator)}\\
                      &=z\cdot \frac{1}{\left(1-\frac{z^{2}}{3!}+\frac{z^{4}}{5!}-...\right)}\\
                      &=z\cdot (1+\frac{z^{2}}{3!}+\frac{z^{4}}{5!}+...)\\
                      &=z+\frac{z^{2}}{3!}+\frac{z^{4}}{5!}+...
                  \end{split}
              \end{equation}
              And again, since the our Laurent expansion contains no $b_{n}$ terms where $(z-z_{0})$ is in the denominator, $z=\pi n$ is a a removable singular point.
        \item For this, we have an isolated singularity for when $e^{2z}-1=0$. To find these points, we rewrite $e^{2z}$ as
              $$e^{2x+2iy}=e^{2x}\cdot e^{2iy}=e^{2x}\left(\cos(2y)+i\sin(2y)\right)$$
              $e^{2x}\left(\cos(2y)+i\sin(2y)\right)=1$ is true for when $x=0$, and $2y=2\pi k$ where $k\in\Z$ which implies that we have singularties for $z=x+iy=i\pi k$. Now Taylor expanding $e^{z}$:
              \begin{equation}
                  \begin{split}
                      \frac{e^{z}-1}{e^{2z}-1}&=\frac{\left(1+z+\frac{z^{2}}{2!}+\frac{z^{3}}{3!}+...\right)-1}{\left(1+2z+\frac{2^{2}z^{2}}{2!}+\frac{2^{3}z^{3}}{3!}+...\right)-1}\\
                      &=\frac{z+\frac{z^{2}}{2!}+\frac{z^{3}}{3!}+...}{2z+\frac{2^{2}z^{2}}{2!}+\frac{2^{3}z^{3}}{3!}+...}\quad\text{(after the $\pm 1$ cancel)}
                      &=\frac{z}{}
                  \end{split}
              \end{equation}
        \item
    \end{enumerate}
\end{proof}


\newpage
\section{Problem 6}
Find the Laurent series for a given function about the point $z=0$ and find the residue at that point.

\begin{enumerate}[label=\textbf{(\alph*)}]
    \item
          \begin{flushleft}
              $\begin{aligned}
                      \frac{e^{z}-1}{z}
                  \end{aligned}$
          \end{flushleft}
    \item
          \begin{flushleft}
              $\begin{aligned}
                      \frac{z}{(\sin(z))^{2}}
                  \end{aligned}$
          \end{flushleft}
    \item
          \begin{flushleft}
              $\begin{aligned}
                      \frac{1}{e^{z}-1}
                  \end{aligned}$
          \end{flushleft}
    \item
          \begin{flushleft}
              $\begin{aligned}
                      \frac{1}{1-\cos(z)}
                  \end{aligned}$
          \end{flushleft}
\end{enumerate}
In \textbf{(c)} and \textbf{(d)} compute only three terms of the Laurent series.

\begin{proof}
    \begin{enumerate}[label=\textbf{(\alph*)}]
        \item
        \item
        \item
        \item
    \end{enumerate}
\end{proof}


\newpage
\section{Problem 7}
Find the residue of $f(z)=\frac{1}{1+z^{n}}$ at the point $z_{0}=e^{i\frac{\pi}{n}}$

\begin{proof}

\end{proof}


\newpage
\section{Problem 8}
Calculate:
\begin{enumerate}[label=\textbf{(\alph*)}]
    \item
          \begin{flushleft}
              $\begin{aligned}
                      \int_{-\infty}^{\infty}\frac{x^{2}}{(1+x^{2})(4+x^{2})}dx
                  \end{aligned}$
          \end{flushleft}
    \item
          \begin{flushleft}
              $\begin{aligned}
                      \int_{-\infty}^{\infty}\frac{dx}{(1+x^{2})^{2}}(=\frac{\pi}{2})
                  \end{aligned}$
          \end{flushleft}
    \item
          \begin{flushleft}
              $\begin{aligned}
                      \int_{-\infty}^{\infty}\frac{x\sin(ax)}{x^{2}+b^{2}}dx(=\pi e^{-ab})
                  \end{aligned}$
          \end{flushleft}
    \item
          \begin{flushleft}
              $\begin{aligned}
                      \int_{-\infty}^{\infty}\frac{\sin(x)}{x}dx(=\pi)
                  \end{aligned}$
          \end{flushleft}
    \item
          \begin{flushleft}
              $\begin{aligned}
                      \int_{0}^{2\pi}\frac{dt}{2+\cos^{2}(t)}dx
                  \end{aligned}$
          \end{flushleft}
\end{enumerate}

\begin{proof}
    \begin{enumerate}[label=\textbf{(\alph*)}]
        \item
        \item
        \item
        \item
        \item
    \end{enumerate}
\end{proof}
\end{document}