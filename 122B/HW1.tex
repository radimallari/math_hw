\documentclass[11pt]{article}
\usepackage{amssymb}
\usepackage{amscd}
\usepackage{amsxtra}
\usepackage{amsmath}
\usepackage{enumitem}
\newcommand{\N}{\mathbb{N}}
\newcommand{\C}{\mathbb{C}}
\newcommand{\R}{\mathbb{R}}
\newcommand{\Z}{\mathbb{Z}}
\newcommand{\Q}{\mathbb{Q}}
\newcommand{\ep}{\varepsilon}
\newcommand{\set}[1]{\left\{ #1\right\}}
\newenvironment{proof}{\noindent{\bf Proof.}}{\hfill $\square$\medskip}

\usepackage[utf8]{inputenc}


\title{Math 122B Homework 1}
\author{Rad Mallari}

\begin{document}
\maketitle

\section{Problem 1}
Let $P(z)$ denote a polynomial of degree $d\geq1$. Define
$$M(r)=\sup_{|z|=r}|P(z)|,\quad r>0$$
Show that both functions
$$r\mapsto M(r),\quad r\mapsto -r^{-d}M(r)$$
are strictly increasing

\begin{proof}

\end{proof}


\section{Problem 2}
Compute the integral
$$\int_{|z|=3}\sin(\frac{1}{z})dz$$

\begin{proof}
We know that the Taylor expansion of $\sin(z)$ about $z=0$ is given by:
$$\sin(z)=z-\frac{z^{3}}{3!}+\frac{z^{5}}{5!}-\dots$$
Therefore
$$\sin(\frac{1}{z})=\frac{1}{z}-\frac{1}{3!z^{3}}+\frac{1}{5!z^{5}}-\dots$$
From this, we see that the $\text{Res}(\sin(\frac{1}{z}),0)=1$, therefore, by the Residue theorem,
$$\int_{|z|=3}\sin(\frac{1}{z})dz=2\pi i\cdot\text{Res}(\sin(\frac{1}{z}),0)=2\pi i$$
\end{proof}


\section{Problem 3}
(Correct statement). Let $n$ be a positive integer and let $f(z)$ be an entire function satisfying the inequality $|f(z)|>|z|^{n}$, for $|z|>1$. Prove that $f$ is a polynomial

\begin{proof}
We know that by \textbf{Theorem 6.11} in the book, we know that if a function $f$ is entire and $f(z)\to\infty$ as $z\to\infty$, then $f$ is a polynomial. 
\end{proof}
\end{document}