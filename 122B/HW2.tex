\documentclass[11pt]{article}
\usepackage{amssymb}
\usepackage{amscd}
\usepackage{amsxtra}
\usepackage{amsmath}
\usepackage{enumitem}
\newcommand{\N}{\mathbb{N}}
\newcommand{\C}{\mathbb{C}}
\newcommand{\R}{\mathbb{R}}
\newcommand{\Z}{\mathbb{Z}}
\newcommand{\Q}{\mathbb{Q}}
\newcommand{\ep}{\varepsilon}
\newcommand{\set}[1]{\left\{ #1\right\}}
\newenvironment{proof}{\noindent{\bf Proof.}}{\hfill $\square$\medskip}

\usepackage[utf8]{inputenc}


\title{Math 122B Homework 2}
\author{Rad Mallari}

\begin{document}
\maketitle

\section{Problem 1}
Prove that for any positive integer $n$,
$$\text{Res}_{0}(1-e^{-z})^{-n}=1$$

\begin{proof}
Letting $f(z)=\frac{1}{(1-e^{-z})^{n}}$, we have singularities at
$$e^{-z}=1\implies z=2\pi ik,\quad k\in\Z$$
By using the using the Residue Theorem, we can compute the integral:
$$\frac{1}{2\pi i}\int_{C}\frac{dz}{(1-e^{-z})^{n}}=\text{Res}_{0}\left(\frac{1}{(1-e^{-z})^{n}}\right)$$
By way of substitution, we can let $w=(1-e^{-z})$ which implies $dw=e^{-z}$, and our equation becomes
$$\frac{1}{2\pi i}\int_{C}\frac{dz}{(1-e^{-z})^{n}}=\frac{1}{2\pi i}\int_{C}\frac{dw}{w^{n}(1-w)}=\text{Res}_{0}\left(\frac{dw}{w^{n}(1-w)}\right)$$
Here it's clear that we have a pole of order $n$ at $w=0$. Solving for the right hand side, we have that:
$$\text{Res}_{0}\left(\frac{dw}{w^{n}(1-w)}\right)=1$$ 
So we can conclude:
$$\frac{1}{2\pi i}\int_{C}\frac{dz}{(1-e^{-z})^{n}}=\text{Res}_{0}\left(\frac{1}{(1-e^{-z})^{n}}\right)=\text{Res}_{0}\left(\frac{dw}{w^{n}(1-w)}\right)=1$$
\end{proof}

\section{Problem 2}
Show that Rouche's Theorem remains valid if the condition: $|f|>|g|$ on $\gamma$ is replaced by:
$|f|\geq|g|$ and $f+g$ does not vanish on $\gamma$.

\begin{proof}
Restating Rouche's Theorem as:
$$|f(z)-g(z)|<|g(z)|$$
Would tell us that there are no zeros in $\gamma$. Furthermore, if $|f(z)|\leq|g(z)|$, then we know that the number of zeros in $f=\alpha\cdot g$ where $\alpha$ is some constant. Replacing $|f|>|g|$ with $|f|\geq|g|$ tell us $|f|>|g|$ or $|f|=|g|$. Since $f+g$ does not vanish on $\gamma$, $f,g$ have an equal number of zeros in $\gamma$. Therefore Rouche's Theorem is valid with the condition $|f|\geq|g|$
\end{proof}

\section{Problem 3}
Show that the function $\sqrt{z^{2}-1}$ can be defined, and it is analytic, on the complex plane
minus the closed interval $[-1,1]$.

\begin{proof}
Letting $f(z)=e^{\frac{1}{2}\log(z^{2}-1)}$, we must prove that $f$ is continuous in $\C\setminus(-\infty, 1)$.
To do this, we use Morera's Theorem which states that if $f$ is continuous on an open set $D$,
and $\int_{\Gamma}f(z)dz=0$ where $\Gamma$ is the boundary of a closed rectangle in $D$, then
$f$ is continuous on $D$. Since $f$ have singularities along $\pm 1$, then $f(z)$ is continuous 
along $\C\setminus(-\infty, 1)$. Furthermore, $f$ is analytic in the domain since
\end{proof}
\end{document}