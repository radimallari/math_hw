\documentclass[11pt]{article}
\usepackage{amssymb}
\usepackage{amscd}
\usepackage{amsxtra}
\usepackage{amsmath}
\usepackage{enumitem}
\newcommand{\N}{\mathbb{N}}
\newcommand{\C}{\mathbb{C}}
\newcommand{\R}{\mathbb{R}}
\newcommand{\Z}{\mathbb{Z}}
\newcommand{\Q}{\mathbb{Q}}
\newcommand{\ep}{\varepsilon}
\newcommand{\set}[1]{\left\{ #1\right\}}
\newenvironment{proof}{\noindent{\bf Proof.}}{\hfill $\square$\medskip}

\usepackage[utf8]{inputenc}


\title{Math 122B Homework 3}
\author{Rad Mallari}

\begin{document}
\maketitle

\section{Problem 1}
Suppose $f$ is a rational function of the form $f=P/Q$ where the polynomials $P$ and $Q$ satisfy
$\deg Q-\deg P\geq 2$. Show that the sum of the residues of $f$ is zero.

\begin{proof}
Letting $R_{0}$ be a radius such that all the poles of $f$ are inside the circle $\gamma:|z|=R_{0}$.
Then by the Residue Theorem we have:
$$\int_{\gamma}f(z)dz=2\pi i\sum_{k=1}^{m}n(\gamma,z_{k})\text{Res}(f;z_{k})$$
for $\gamma>R_{0}$, and so $n(\gamma,z_{k})=1$. Now we let $Q(z)=\alpha_{0}+\alpha_{1}z+\dots+\alpha_{n}z^{n}$,
and $P(z)=\beta_{0}+\beta_{1}z+\dots+\beta_{m}z^{m}$, and divide out $\gamma$ to $|P(z)|$ and $|Q(z)|$ giving us:
$$\left|\int_{\gamma}f(z)dz\right|\leq\left(\frac{\gamma^{m}}{\gamma^{n}}\frac{\beta_{m}}{\alpha_{n}}\right)4\pi \gamma$$
From here we see that if $n-(m+1)\geq 1\Rightarrow n-m\geq 2$, then $\int_{C_{R}}f(z)dz\rightarrow 0$
as $\gamma$ tends to infinity. Therefore we can conclude that if $\deg Q-\deg P\geq 2$ would result in
$\int_{C_{R}}f(z)dz=0=2\pi i\text{Res}(f;z_{k})$.
\end{proof}



\section{Problem 2}
Evaluate the following sums:
$$\sum_{n=1}^{\infty}\frac{1}{n^{2}+1}$$
$$\sum_{n=1}^{\infty}\frac{1}{n^{4}}$$
$$\sum_{n=1}^{\infty}\frac{(-1)^{n}}{n^{2}+1}$$


\begin{proof}
    \begin{enumerate}[label=\textbf{(\alph*)}]
        \item We can use
                $$\sum_{\substack{n=-\infty\\n\neq z_{k}}}^{\infty}f(n)=-\sum_{k}\text{Res}(f(z)\pi\cot\pi z,z_{k})$$
                where $C_{N}$ is a simple closed contour, $n$ are integers inside $C_{N}$, $z_{k}$ are poles of $f$. In our case, then, we have:
                $$2\sum_{n=1}^{\infty}\frac{1}{n^{2}+1}=-\sum_{k=1}^{3}\text{Res}\left(\frac{\pi\cot(\pi z)}{1+z^{2}},z_{k}\right)$$
                Before solving for the residues, we know that $\cot z$ has a Laurent expansion about $0$ given by:
                $$\cot z=\frac{1}{z}-\frac{z}{3}-\frac{1}{45}z^{3}+...$$
                Using this, we can expand $\frac{\pi\cot\pi z}{1+z^{2}}$ about $0$ and get:
                \begin{equation}
                    \begin{split}
                        \frac{\pi\cot\pi z}{1+z^{2}}&=\frac{1}{1+z^{2}}\left(\frac{1}{z}-\frac{\pi^{2}}{3}-\frac{\pi^{4}z^{3}}{45}+...\right)\\
                        &=\frac{1}{z+z^{2}}-\frac{\pi^{2}z}{3(1+z^{2})}-\frac{\pi^{4}z^{3}}{45(1+z^{2})}-...\\
                        &=\frac{1}{z}+\frac{-1}{1+z}-\frac{\pi^{2}z}{3(1+z^{2})}-\frac{\pi^{4}z^{3}}{45(1+z^{2})}-...
                    \end{split}
                \end{equation}
                This implies that $\text{Res}(\frac{\pi\cot(\pi z)}{1+z^{2}},0)=1$ giving us our residue at $z_{1}=0$. Now for the poles at $z=\pm i$, since they are simple poles, we know that the residues is given by:
                $$\text{Res}(f,z)=\lim_{z\to z_{0}}(z-z_{0})f(z)$$
                This tells us that for $z_{2}=i$ and $z_{3}=-i$:
                $$\text{Res}(f,z_{2})=\lim_{z\to i}\frac{\pi\cot(i\pi)}{z+i}=\frac{\pi\cot(i\pi)}{2i}$$
                $$\text{Res}(f,z_{3})=\lim_{z\to -i}\frac{\pi\cot(-i\pi)}{z-i}=\frac{\pi\cot(i\pi)}{2i}$$
                Therefore,
                $$2\sum_{n=1}^{\infty}\frac{1}{n^{2}+1}=-\left(1+\frac{\pi\cot(\pi i)}{i}\right)$$
                To simplify the second term, we recall
                $$\cot(z)=\frac{ie^{iz}+ie^{-iz}}{e^{iz}-e^{-iz}}$$
                Substituting this to our equation yields:
                \begin{equation}
                    \begin{split}
                        2\sum_{n=1}^{\infty}\frac{1}{n^{2}+1}&=-\left(1+\frac{\pi i\left(\frac{e^{-\pi}+e^{\pi}}{e^{-\pi}-e^{\pi}}\right)}{i}\right)\\
                        &=-\frac{1}{2}\left(1+\pi\frac{e^{-\pi}+e^{\pi}}{e^{-\pi}-e^{\pi}}\right)
                    \end{split}
                \end{equation}
        \item We can rewrite this summation as:
        $$\sum_{n=1}^{\infty}\frac{1}{n^{4}}=\frac{1}{2}\sum_{\substack{n=-\infty\\n\neq0}}^{\infty}\frac{1}{n^{4}}$$
        Then by the same theorem as \textbf{(a)}, we have that
        $$\sum_{n=1}^{\infty}\frac{1}{n^{4}}=-\frac{1}{2}\text{Res}\left(\frac{\pi\cot\pi z}{z^{4}},0\right)$$
        We know that we have a singularity at $z=0$ and the Laurent expansion about $z=0$ for $\frac{\pi\cot\pi z}{z^{4}}$ is given by:
        $$\frac{\pi\cot(\pi z)}{z^{4}}=\frac{1}{z^{5}}-\frac{\pi^{2}}{3z^{3}}-\frac{\pi^{4}}{45z}-...$$
        Therefore,
        $$\text{Res}\left(\frac{\pi\cot\pi z}{z^{4}},0\right)=-\frac{\pi^{4}}{45}$$
        and we can conclude that 
        $$\sum_{n=1}^{\infty}\frac{1}{n^{4}}=\frac{\pi^{4}}{90}$$
        \item Lastly, by Residue Theorem, we know
        $$\sum_{\substack{n=-\infty\\n\neq z_{k}}}^{\infty}(-1)f(n)=-\sum_{k}\text{Res}(\pi f(z)\csc\pi z, z_{k})$$
        Then in this case,
        $$1+2\sum_{n=1}^{\infty}\frac{(-1)^{n}}{n^{2}+1}=-\sum_{z=\pm i}\text{Res}\left(\frac{\pi}{(\sin\pi z)(z^{2}+1)},z_{k}\right)$$
        Finding our residues for $z_{1}=i$, $z_{2}=-i$, we have:
        \begin{equation}
            \begin{split}
                \text{Res}\left(\frac{\pi}{(\sin\pi z)(z^{2}+1)},z_{1}\right)&=(z-i)\lim_{z\to i}\frac{\pi}{(\sin\pi z)(z+i)(z-i)}\\
                &=\frac{\pi}{(\sin i\pi)(2i)}
            \end{split}
        \end{equation}
        and
        \begin{equation}
            \begin{split}
                \text{Res}\left(\frac{\pi}{(\sin\pi z)(z^{2}+1)},z_{2}\right)&=(z+i)\lim_{z\to-i}\frac{\pi}{(\sin\pi z)(z+i)(z-i)}\\
                &=\frac{\pi}{(\sin i\pi)(2i)}
            \end{split}
        \end{equation}
        So we know that:
        \begin{equation}
            \begin{split}
            -\sum_{z=\pm i}\text{Res}\left(\frac{\pi}{(\sin\pi z)(z^{2}+1)},z_{k}\right)&=\frac{\pi}{i(\sin i\pi)}\\
            &=\frac{2\pi}{e^{\pi}-e^{-\pi}}\quad\left(\text{by }\sin x=\frac{e^{ix}-e^{-ix}}{2i}\right)
            \end{split}
        \end{equation}
        Therefore, we can conclude that
        $$\sum_{n=1}^{\infty}\frac{(-1)^{n}}{n^{2}+1}=-\frac{1}{2}+\frac{2\pi}{e^{\pi}-e^{-\pi}}$$
    \end{enumerate}

\end{proof}


\section{Problem 3}
Let $U$ be an open set of the complex plane. Find conditions on $U$ assuring that:
\begin{enumerate}[label=\textbf{(\alph*)}]
    \item The function $z\mapsto z^{2}$, $z\in U$, is one to one.
    \item The function $z\mapsto \cos(z)$, $z\in U$, is one to one.
\end{enumerate}

\begin{proof}
    \begin{enumerate}[label=\textbf{(\alph*)}]
        \item We define a function $f$ to be one to one if for all arbitrary $z_{1}, z_{2}$ in some region
        $D$, $f(z_{1})\neq f(z_{2})$. This condition only fails for $z=0$, so any open set $U\subset\C\setminus0$ is valid.
        \item 
    \end{enumerate}
\end{proof}
\end{document}