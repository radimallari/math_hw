\documentclass[11pt]{article}
\usepackage{amssymb}
\usepackage{amscd}
\usepackage{amsxtra}
\usepackage{amsmath}
\usepackage{enumitem}
\newcommand{\N}{\mathbb{N}}
\newcommand{\C}{\mathbb{C}}
\newcommand{\R}{\mathbb{R}}
\newcommand{\Z}{\mathbb{Z}}
\newcommand{\Q}{\mathbb{Q}}
\newcommand{\ep}{\varepsilon}
\newcommand{\set}[1]{\left\{ #1\right\}}
\newenvironment{proof}{\noindent{\bf Proof.}}{\hfill $\square$\medskip}

\usepackage[utf8]{inputenc}


\title{Math 122B Homework 4}
\author{Rad Mallari}

\begin{document}
\maketitle

\section{Problem 1}
Verify directly that the function $f(z)=z^{2}+z+1$ is injective in a neighborhood of the point $z=0$.

\begin{proof}
A function $f(z)$ is locally one to one around $z_{0}$ if it is analytic at $z_{0}$ and $f'(z_{0})\neq0$.
Clearly $f(z)$ is analytic at $0$, meanwhile $f'(0)=0^{2}+0+1\neq0$, therefore $f(z)$ is injective about $z=0$.
\end{proof}


\section{Problem 2}
Find the image of the square $|\mathfrak{R}z|<1$, $|\mathfrak{I}z|<1$, under the exponential map $z\mapsto e^{z}$.

\begin{proof}
Letting $z=x+iy$, we can rewrite our exponential map as $e^{z}=e^{x}e^{iy}$. Defining $z$ this way,
we know that $\mathfrak{R}z=x$, meanwhile $\mathfrak{I}z=y$. Therefore, we have a square region
$-1\leq x\leq 1$ and $-1\leq y\leq 1$. Now we can rewrite our mapping as $w=\rho e^{i\phi}$, where
$\rho=e^{x}$ and $\phi=y$. This tells us that our mapping is a circle around $0$ with radius $1$.
\end{proof}


\section{Problem 3}
Describe a conformal map of the infinite band $-2<\mathfrak{R}z<1$, onto the unit disk.

\begin{proof}
    
\end{proof}


\section{Problem 4}
Find all conformal mappings $h(z)$ from the upper-half plane to itself satisfying $f(i)=i$.

\begin{proof}
    
\end{proof}
\end{document}