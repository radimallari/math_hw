\documentclass[11pt]{article}
\usepackage{amssymb}
\usepackage{amscd}
\usepackage{amsxtra}
\usepackage{amsmath}
\usepackage{enumitem}
\newcommand{\N}{\mathbb{N}}
\newcommand{\C}{\mathbb{C}}
\newcommand{\R}{\mathbb{R}}
\newcommand{\Z}{\mathbb{Z}}
\newcommand{\Q}{\mathbb{Q}}
\newcommand{\ep}{\varepsilon}
\newcommand{\set}[1]{\left\{ #1\right\}}
\newenvironment{proof}{\noindent{\bf Proof.}}{\hfill $\square$\medskip}

\usepackage[utf8]{inputenc}


\title{Math 122B Homework 6}
\author{Rad Mallari}

\begin{document}
\maketitle

\section{Problem 1}
Let $a>0$ be a positive parameter. Find the image of the exterior of the unit circle by the comformal map
$$w=az+\frac{1}{z}$$

\begin{proof}
Letting $z=re^{it}$, we get that $w=a(re^{it})+\frac{1}{re^{it}}$. Since $e^{it}=\cos(t)+i\sin(t)$, we can rewrite our comformal map as:
\begin{equation}
    \begin{split}
        w&=ar(\cos(t)+i\sin(t))+r(\cos(-t)+i\sin(-t))\\
        &=r\left[a(\cos(t)+i\sin(t))+\cos(t)-i\sin(t)\right]\\
        &=r\cos(t)(a+1)+ir\sin(t)(a-1)\\
        &=r\left[\cos(a+1)+i\sin(a-1)\right]
    \end{split}
\end{equation}
Which describes an ellipse.
\end{proof}


\section{Problem 2}
Let $U$ be a simply connected domain, different than the entire complex plane. Let $z_{0}\in U$. Let $G$ denote the class of all analytic functions $g:U\to D$ satisfying $g'(z_{0})>0$. Show that
$$\sup_{g\in G}g'(z_{0})<\infty$$
and that the supremum is attained by a function $f\in G$. Prove $f$ is one-to-one.

\begin{proof}
    
\end{proof}


\section{Problem 3}
Let $u$ be a harmonic function. Show that $u^{2}$ is harmonic if and only if $u$ is constant.

\begin{proof}
$u^{2}$ implies that 
\begin{equation}
    \begin{split}
        (u^{2})_{xx}+(u^{2})_{yy}&=2u(u_{xx}+u_{yy})+2(u^{2}_{x}+u^{2}_{y})
    \end{split}
\end{equation}
Since it is harmonic, $(u^{2})_{xx}+(u^{2})_{yy}=0$, therefore $u_{xx}+u_{yy}=0$ and
\begin{equation}
    \begin{split}
        u^{2}_{x}+u^{2}_{y}&=0\\
        u_{x}=u_{y}=0
    \end{split}
\end{equation}
And we can san see that $u$ is constant.
\end{proof}


\section{Problem 4}
Suppose the function $f=u+iv$ is analytic. Prove $uv$ is a harmonic function. Give example of two harmonic functions $U,V$ with the properties that $UV$ is not harmonic.

\begin{proof}
Using the definition of $f$, we have
$$f^{2}=u^{2}+2iuv-v^{2}\quad\text{and}\quad \frac{f^{2}}{2}=\frac{u^{2}}{2}+iuv-\frac{v^{2}}{2}$$
And that the imaginary part of $\frac{^{2}}{2}$ is $uv$, while the real part is $u^{2}-v^{2}$. By 
\textbf{Theorem 16.2}, the real and imaginary parts of a harmonic function are harmonic, therefore $uv$ is harmonic.

Letting $u=x,v=x$ we have that $uv=x^{2}$. Since $u_{xx}=2$ and $v_{xy}=0$, we see that $uv$ is \textit{not} harmonic.
Another is example is by letting $u=x$ and $v=x^{2}-y^{2}$. This implies that $uv=x^{3}-xy^{2}$, and $v_{xx}=3$ while $v_{xy}=-2y$ so $uv$ is \textit{not} harmonic.
\end{proof}
\end{document}