\documentclass[11pt]{article}
\usepackage{amssymb}
\usepackage{amscd}
\usepackage{amsxtra}
\usepackage{amsmath}
\usepackage{enumitem}
\newcommand{\N}{\mathbb{N}}
\newcommand{\C}{\mathbb{C}}
\newcommand{\R}{\mathbb{R}}
\newcommand{\Z}{\mathbb{Z}}
\newcommand{\Q}{\mathbb{Q}}
\newcommand{\ep}{\varepsilon}
\newcommand{\set}[1]{\left\{ #1\right\}}
\newenvironment{proof}{\noindent{\bf Proof.}}{\hfill $\square$\medskip}

\usepackage[utf8]{inputenc}


\title{Math 119A Homework 1}
\author{Rad Mallari}

\begin{document}
\maketitle

Notation. Let $D_{\infty}[0,1]$ be the set of infinitely differentiable, real-valued functions on the real interval $[0,1]$.
Let $L=\set{f\in D_{\infty}[0,1]:f(0)=0}$

\section{Problem 1}
Does $\delta(f)(s)=f^{\prime}(s)$ define an injective and surjective function from $L\rightarrow L$?

\begin{proof}
In order to prove injectivity we must show that for any values, in this case functions $f_{1},f_{2}$, implies
$f_{1}=f_{2}$ if we have $\delta(f_{1})(s)=\delta(f_{2})(s)$. Taking an integration of both sides yield
\begin{equation}
    \begin{split}
        \int\delta(f_{1})(s)&=\int\delta(f_{2})(s)\\
        f_{1}(s)=f_{2}(s)+c_{1}\quad&\text{and}\quad f_{2}(s)=f_{1}(s)+c_{2}.
    \end{split}
\end{equation}
We know that we must must $f(0)=0$ for all $f\in D_{\infty}[0,1]$ which implies that $f_{1}(0)=0+c_{1}$ and $f_{2}(0)=0+c_{2}$
which implies $f_{1}(0)=f_{2}(0)=0$ which tells us that $c_{1}=-c_{2}$. Therefore, by substituting the results of our constants 
to the result of our integral in $(1)$ shows us that $f_{1}(s)=f_{2}(s)$ which satisfies our requirements for injectivity.

To prove surjectivity, we must show that for any arbitrary function $g$, there exists and element $f\in L$ such that 
$\delta (f)(s)=g$. Taking the integral of this yields $\int\delta (f)(s)=g\implies f(s)=\int g(s)ds$. We know that $g(0)=0$ 
since $g\in L$ therefore, our integral becomes $f(0)=\int g(0)ds=0$, satisfying $f(0)=0$. This proves that for any $g\in L$, 
we meet our surjectivity requirements.
\end{proof}

\newpage
\section{Problem 2}
Same question for $\iota(f)(t)=\int_{0}^{t}f(s)ds$.

\begin{proof}
Taking the same requierments as Problem 1, we take some value $f_{1},f_{2}$ and want to show that if 
$\iota f_{1}(t)=\iota f_{2}(t)$ this implies that $f_{1}=f_{2}$. By definition we have that
$$\int_{0}^{t}f_{1}(s)ds=\int_{0}^{t}f_{2}(s)ds$$
Which taking the derivative with respect to $t$ we get
\begin{equation}
    \begin{split}
        \frac{d}{dt}\left(\int_{0}^{t}f_{1}(s)ds\right)&=\frac{d}{dt}\left(\int_{0}^{t}f_{2}(s)ds\right)\\
        f_{1}(t)&=f_{2}(t)
    \end{split}
\end{equation}
This shows that $f_{1}(t)=f_{2}(t)$ for all $t\in[0,1]$.

Similar to (1), we can show that $\iota(f)(t)$ is surjective by taking arbitrary element $g\in L$ such that there exists
$f\in L$ that satisfies $\iota f(t)=g(t)$ for all $t\in[0,1]$. Taking $\iota f(t)=g(t)$, we have the following
$$f(t)=\int_{0}^{1}g(s)ds$$
Using our given $f(0)=0$, we have
\begin{equation}
    \begin{split}
        f(0)&=\int_{0}^{1}g(0)ds\\
        f(0)&=0
    \end{split}
\end{equation}


\end{proof}

\newpage
\section{Problem 3}
Are $\iota$ and $\delta$ inverse functions? I.e. does $\iota(\delta(f))=\delta(\iota(f))=f$ for all $f\in L$?

\begin{proof}
Yes, $\iota$ and $\delta$ inverse functions since by definition of derivatives $\delta(\int_{0}^{t}f(s)ds)=f(s)$.
Likewise $\iota(f^{\prime}(s))=f(s)$ by definition of integrals.
\end{proof}

\section{Problem 4}
Look at the solutions to some problems in Chapter 1 (on p. 343 of the text).
Prove or disprove that these solutions are correct. Be sure to explain what you are doing in terms of the meaning of the words
in the problems. Of course this problem is really several problems.

\begin{proof}
For problem 2(a), the matrix $A$ is correct as taking $x^{\prime}(0)$, we get $(k_{1}e^{0},k_{2}e^{0},k_{3}e^{0}=(k_{1},k_{2},k_{3})$.
Which satisfies our differential equation and our initial value requirement of $x(0)=(k_{1},k_{2},k_{3})$. 
Meanwhile, 2(b) fails since $x^{\prime}(0)$ yields $(k_{1}e^{0},-2k_{2}e^{0},0)=(k_{1},k_{2},0)$ which fails our initial value 
requirement. Similar to 2(a), our solution satisfies both the differential equation and plugging in $x^{\prime}(0)$ gives us 
the correct initial value.
\end{proof}

\section{Problem 5}
Let $A$ be as in
$$\begin{bmatrix}
    0&-1&0\\
    1&0&0\\
    0&0&-\frac{1}{2}
\end{bmatrix}$$
Find constants $a,b,c$ such that the curve $t\rightarrow (a\cos(t),b\sin(t),ce^{-\frac{t}{2}})$ is a solution to $x^{\prime}=Ax$
with $x(0)=(1,0,3)$.

\begin{proof}
The solution to our differential equation can be seen if by rearranging switching row 1 and row 2 to get
$$\begin{bmatrix}
    1&0&0\\
    0&-1&0\\
    0&0&-\frac{1}{2}
\end{bmatrix}$$
where it is obvious that our solution is $x=(k_{1}e^{t},k_{2}e^{-t},k_{3}e^{-\frac{1}{2}t})$.
\end{proof}

\newpage
\section{Problem 6}
Find two different matrices $A,B$ such that the curve
$$x(t)=(e^{t},2e^{2t},4e^{2t})$$
satisfies both the differential equations
$$x^{\prime}Ax\quad\text{and}\quad x^{\prime}=Bx$$

\begin{proof}
Letting $A$ be 
$$\begin{bmatrix}
    0&0&\frac{1}{8}\\
    2&0&0\\
    0&2&0
\end{bmatrix}$$
we have the following
$$\begin{bmatrix}
    e^{t}\\
    4e^{2t}\\
    8e^{2t}
\end{bmatrix}=\begin{bmatrix}
    0&0&\frac{1}{8}\\
    4&0&0\\
    0&2&0
\end{bmatrix}
\begin{bmatrix}
    e^{t}\\
    2e^{2t}\\
    4e^{2t}
\end{bmatrix}$$
Likewise taking $B$ to be
$$\begin{bmatrix}
    1&0&0\\
    0&2&0\\
    4&0&1
\end{bmatrix}
$$
We satisfy the same differential equation $x^{\prime}=Bx$ as shown in the following
$$\begin{bmatrix}
    e^{t}\\
    4e^{2t}\\
    8e^{2t}
\end{bmatrix}=
\begin{bmatrix}
    1&0&0\\
    0&2&0\\
    4&0&1
\end{bmatrix}
\begin{bmatrix}
    e^{t}\\
    2e^{2t}\\
    4e^{2t}
\end{bmatrix}$$
\end{proof}
\end{document}