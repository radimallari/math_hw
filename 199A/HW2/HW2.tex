\documentclass[11pt]{article}
\usepackage{amssymb}
\usepackage{amscd}
\usepackage{amsxtra}
\usepackage{amsmath}
\usepackage{enumitem}
\newcommand{\N}{\mathbb{N}}
\newcommand{\C}{\mathbb{C}}
\newcommand{\R}{\mathbb{R}}
\newcommand{\Z}{\mathbb{Z}}
\newcommand{\Q}{\mathbb{Q}}
\newcommand{\ep}{\varepsilon}
\newcommand{\set}[1]{\left\{ #1\right\}}
\newcommand{\norm}[1]{\left\lVert#1\right\rVert}
\newenvironment{proof}{\noindent{\bf Proof.}}{\hfill $\square$\medskip}

\usepackage[utf8]{inputenc}


\title{Math 119A Homework 2}
\author{Rad Mallari}

\begin{document}
\maketitle

\section{Problem 1}
A particle of mass \textit{m} moves in the plane $\R^{2}$ under the influence of an elastic band tying it to the origin.
The length of the band is negligible. Hooke's law states taht the force on the particle is always directed toward
the origin and is proportional to the distance from the origin. Write the force field and verify that it is
conservative and central. Write the equation $F=ma$ for this case and solve it. Verify that for \textit{``most"} intial
conditions the particle moves in an ellipse

\begin{proof}
    
\end{proof}

\section{Problem 2}
Which of the following force fields on $\R^{2}$ are conservative?
\begin{enumerate}[label=\textbf{(\alph*)}]
    \item $F(x,y)=(-x^{2},-2y^{3})$
    \item $F(x,y)=(x^{2}-y^{2},2xy)$
    \item $F(x,y)=(fx,y)$
\end{enumerate}

\begin{proof}
    \begin{enumerate}[label=\textbf{(\alph*)}]
        \item
        \item 
        \item
    \end{enumerate}
\end{proof}

\section{Problem 3}
Consider the case of a particle in a gravitational field moving directly away from the origin at time $t=0$. Dicuss its 
motion. Under what initial conditions does it eventually reverse direction?

\begin{proof}
    
\end{proof}

\section{Problem 4}
Using the norms $\norm{f}_{\infty}=\sup\set{|f(s)|:s\in[0,1]}$ and $\norm{f}_{1}=\set{\int_{0}^{1}|f(s)ds|}$ on 
$D_{\infty}[0,1]$ as domain or range, is either $\iota$ or $\delta$ a bounded function?\\
There are really 4 questions here.
\begin{enumerate}[label=\textbf{(\alph*)}]
    \item ``Does there exist a real number $M$ such that if $f\in D_{1}[0,1]$,$\norm{f}_{1}=1$, then 
    $\norm{\iota(f)}_{\infty}\leq M$''
    \item ``Does there exist a real number $M$ such that if $f\in D_{\infty}[0,1]$,$\norm{f}_{\infty}=1$, then 
    $\norm{\iota(f)}_{\infty}\leq M$''
    \item ``Does there exist a real number $M$ such that if $f\in D_{\infty}[0,1]$,$\norm{f}_{1}=1$, then 
    $\norm{\iota(f)}_{1}\leq M$''
    \item ``Does there exist a real number $M$ such that if $f\in D_{\infty}[0,1]$,$\norm{f}_{\infty}=1$, then 
    $\norm{\iota(f)}_{1}\leq M$''
\end{enumerate}
Then do 4 more by replacing $\iota$ with $\delta$. Eight problems in all.\\
For each of these eight problems, you must either specify an $M$ (e.g $M=13$) and prove the desired inequality, OR
you must assume (proof by contradiction) that some unspecified number $M$ works and specify an $f\in D_{\infty}[0,1]$
(depending on $M$) for which this inequality fails.

\begin{proof}
    \begin{enumerate}[label=\textbf{(\alph*)}]
        \item
        \item 
        \item 
        \item 
    \end{enumerate}
\end{proof}
\end{document}