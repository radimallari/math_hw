\documentclass[12pt]{article}
\usepackage{amssymb}
\usepackage{amscd}
\usepackage{amsxtra}
\usepackage{enumitem}
\newcommand{\N}{\mathbb{N}}
\newcommand{\C}{\mathbb{C}}
\newcommand{\R}{\mathbb{R}}
\newcommand{\Z}{\mathbb{Z}}
\newcommand{\Q}{\mathbb{Q}}
\newcommand{\ep}{\varepsilon}
\newcommand{\set}[1]{\left\{ #1\right\}}
\newenvironment{proof}{\noindent{\bf Proof.}}{\hfill $\square$\medskip}

\usepackage[utf8]{inputenc}


\title{Math 122A Homework 3}
\author{Rad Mallari}

\begin{document}
\maketitle

\section{Problem 1}
Prove that $f(z)=\lvert z\rvert^{2}$ is not analytic in any open set $A\subset\C$.

\begin{proof}
Letting $z=x+iy$, we know that $\lvert z\rvert^{2}=x^{2}+y^{2}+i0$. Taking the real part of $f(z)$ as $u$ we have that $u(x,y)=x^{2}+y^{2}$ and the imaginary as $v$, $v(x,y)=0$. For $f(z)$ to be analytic, we know that $\forall z\in A$, $f'(z)$ must exist. So in order to check this, we take $\frac{\partial u}{\partial x}=2x$, $\frac{\partial u}{\partial y}=2y$, $\frac{\partial v}{\partial x}=0$, $\frac{\partial v}{\partial y}=0$. Now by Cauchy Riemann Equations, $f(z)$ would be differentiable for all $z\in A$ if $\frac{\partial u}{\partial x}=\frac{\partial v}{\partial y}$, and $\frac{\partial u}{\partial y}=-\frac{\partial v}{\partial x}$, but using this we have that $2x\neq 0$ and $2y\neq -0$. Therefore, $f(z)$ is not analytic in any open set.
\end{proof}


\section{Problem 2}
Let $A\subset\C$ be an open set. Assume that $f:A\rightarrow\C$ with $f(z)=f(x+iy)=u(x,y)+iv(x,y)$ is analytic on $A$, i.e. $f'(z)$ exists for any point $z\in A$. By using change of variables, deduce the Cauchy-Riemann equations in polar coordinates
$$\frac{\partial u}{\partial r}=\partial_{r}u=\frac{1}{r}\partial_{\theta}v=\frac{\partial v}{\partial\theta},\qquad\partial_{r}v=-\frac{1}{r}\partial_{\theta}u.$$

\begin{proof}
For $f$ to be analytic, we must satisfy Cauchy-Riemann equations, similar to Problem 1. So, letting $z=x+iy=r(\cos\theta+i\sin\theta)$, and taking $u(x,y)$ be the real term and $v(x,y)$ be the imaginary term, we have that
$$\frac{\partial u}{\partial r}=\frac{\partial u}{\partial x}\cos\theta+\frac{\partial u}{\partial y}\sin\theta$$
And by the Cauchy-Riemann we get
$$\frac{\partial u}{\partial r}=\frac{1}{r}(\frac{\partial v}{\partial y}r\cos\theta-\frac{\partial v}{\partial x}r\sin\theta)=\frac{1}{r}(\frac{\partial v}{\partial\theta})$$
Similarly, we using the chain rule,
$$\frac{\partial v}{\partial r}=\frac{\partial v}{\partial x}\cos\theta+\frac{\partial v}{\partial y}\sin\theta$$
$$\frac{\partial v}{\partial r}=-\frac{1}{r}(\frac{\partial u}{\partial y}r\cos\theta-\frac{\partial u}{\partial x}r\sin\theta)=-\frac{1}{r}(\frac{\partial u}{\partial\theta})$$
Giving us that:
$$\frac{\partial u}{\partial r}=\partial_{r}u=\frac{1}{r}\partial_{\theta}v=\frac{\partial v}{\partial\theta},\qquad\partial_{r}v=-\frac{1}{r}\partial_{\theta}u.$$
\end{proof}


\section{Problem 3}
\underline{DEFINITION} Let $A\subset\R^{2}$ be an open set. A function $h:A\subset\R^{2}\rightarrow\R^{2}$ is said to be harmonic if:
\begin{enumerate}
    \item $h$ is twice differentiable in each variable in any point $(x,y)\in A$
    \item $\partial_{x}^{2}h(x,y)+\partial_{y}^{2}h(x,y)=0$ for any point $(x,y)\in A$
\end{enumerate}
Given any $f:A\rightarrow\C$ with $f(z)=f(x+iy)=u(x,y)+iv(x,y)$ analytic on $A$, and assuming that $u,v$ are twice differentiable in each variable at any point $(x,y)\in A$, prove that $u(x,y)$ and $v(x,y)$ are harmonic.

\begin{proof}
Since $f$ is analytic, we know that
$$\frac{\partial u}{\partial x}=\frac{\partial v}{\partial y}\quad\text{and}\quad\frac{\partial u}{\partial y}=-\frac{\partial v}{\partial x}$$
We also know that $u$, and $v$ are twice differentiable so
$$\frac{\partial^{2}u}{\partial x^{2}}=\frac{\partial^{2}v}{\partial xy},\quad\frac{\partial^{2}u}{\partial xy}=\frac{\partial^{2}v}{\partial y^{2}}\quad\text{and}\quad\frac{\partial^{2}u}{\partial y^{2}}=-\frac{\partial^{2}v}{\partial xy},\quad \quad\frac{\partial^{2}u}{\partial yx}=-\frac{\partial^{2}v}{\partial x^{2}}$$
\end{proof}


\section{Problem 4}
In each of the following cases check if that given function $u=u(x,y)$ is harmonic (and in which domain), and if this is the case, find $v=v(x,y)$ such that $f:A\rightarrow\C$ with $f(z)=f(x+iy)=u(x,y)+iv(x,y)$ is analytic on $A$, i.e. $v=v(x,y)$ is the harmonic conjugate of $u(x,y)$ (unique up to a constant). Wirte $f$ as function of $z$, i.e. $f(z)$.
\begin{enumerate}[label=\textbf{(\alph*)}]
    \item $u(x,y)=e^{y}sin(x)$
    \item $u(x,y)=(x+y)^{2}$
    \item $u(x,y)=x+y^{2}$
    \item $u(x,y)=\ln(x^{2}+y^{2})$
    \item $u(x,y)=\tan^{-1}(\frac{y}{x})$
\end{enumerate}

\begin{proof}
\begin{enumerate}[label=\textbf{(\alph*)}]
    \item Taking the derivative twice, with respect to $x$ and $y$ we have that $\frac{\partial^{2}u}{\partial x^{2}}=-e^{y}sin(x)$ and $\frac{\partial^{2}u}{\partial y^{2}}=e^{y}sin(x)$ and we get that
    $$\frac{\partial^{2}u}{\partial x^{2}}+\frac{\partial^{2}u}{\partial y^{2}}=-e^{y}sin(x)+e^{y}sin(x)=0$$
    So $u(x,y)$ is a harmonic. To find $v(x,y)$, such that $f(z)$ is analytic, we use the Cauchy-Riemann equations
    $$\frac{\partial u}{\partial x}=\frac{\partial v}{\partial y}=e^{y}\cos(x)\quad\text{and}\quad\frac{\partial u}{\partial y}=-\frac{\partial v}{\partial x}=e^{y}\sin(x)$$
    Taking the integrals we get that $v=e^{y}\sin(x)+c_{0}$ and $v=e^{y}\sin(x)+c_{1}$, which implies that $v(x,y)=e^{y}\sin(x)+c$ where $c\in\R$. So $$f(x,y)=u(x,y)+iv(x,y)=e^{y}\sin(x)+i(e^{y}\cos(x)+c)$$
    Writing $f$ as a function of $z$,
    \item Taking the derivative for each variable, we have that $\frac{\partial^{2}u}{\partial x^{2}}=2\cdot(1+y)$ and $\frac{\partial^{2}u}{\partial y^{2}}=2\cdot(1+x)$. This gives us that
    $$\frac{\partial^{2}u}{\partial x^{2}}+\frac{\partial^{2}u}{\partial y^{2}}=2(2+x+y)\neq0$$
    Therefore, $u(x,y)$ is not harmonic.
    \item Similar to \textbf{(b)}, we have that $\frac{\partial^{2}u}{\partial x^{2}}=0$ and $\frac{\partial^{2}u}{\partial y^{2}}=2$ giving us that
    $$\frac{\partial^{2}u}{\partial x^{2}}+\frac{\partial^{2}u}{\partial y^{2}}=0+2\neq0$$
    Therefore, $u$ in this case is not harmonic.
    \item Again, we have $\frac{\partial^{2}u}{\partial x^{2}}=-\frac{4x}{(x^{2}+y^{2})^{2}}+\frac{2}{x^{2}+y^{2}}$ and $\frac{\partial^{2}u}{\partial y^{2}}=-\frac{4y}{(x^{2}+y^{2})^{2}}+\frac{2}{x^{2}+y^{2}}$, so $u$ is not harmonic in this case because $\frac{\partial^{2}u}{\partial x^{2}}+\frac{\partial^{2}u}{\partial y^{2}}\neq0$
    \item Finally, using the same method, we have
    $$\frac{\partial^{2}u}{\partial x^{2}}=\frac{-1}{x^{2}+y^{2}}+\frac{2xy}{(x^{2}+y^{2})^{2}}$$
    and
    $$\frac{\partial^{2}u}{\partial y^{2}}=\frac{4y}{x(x+\frac{y^{2}}{x})^{2}}$$
    which, again, $\frac{\partial^{2}u}{\partial x^{2}}+\frac{\partial^{2}u}{\partial y^{2}}\neq0$, therefore $u$ is not harmonic.
\end{enumerate}
\end{proof}


\newpage
\section{Problem 5}
For which values of the real constants $a,b,c,d\in\R$ is
$$u(x,y)=ax^{3}+bx^{2}y+cxy^{2}+dy^{3}$$
harmonic? In those cases, find its harmonic conjugate $v=v(x,y)$.\\
Write the analytic function $f(z)=f(x+iy)=u(x,y)+iv(x,y)$ of $z$, i.e. $f(z)$.

\begin{proof}
For $u$ to be harmonic, we want
$$
\frac{\partial^{2}u}{\partial x^{2}}+\frac{\partial^{2}u}{\partial y^{2}}=6ax+2by+2cx+6dy=0
$$
Grouping the terms gives us
$$x(6a+2c)+y(6d+2b)=0$$
Which implies that $b=-3d$ and $c=-3a$. So, $$u(x,y)=ax^{3}-3dx^{2}y-3axy^{2}+dy^{3}$$
Now to find $v$ we use the Cauchy-Riemann equations to get
$$\frac{\partial u}{\partial x}=\frac{\partial v}{\partial y}=3ax^{2}-6dxy-3ay^{2}$$
and
$$-\frac{\partial u}{\partial y}=\frac{\partial v}{\partial x}=3dx^{2}+6axy-3dy^{2}$$
Taking the integrals yields
$$v(x,y)=3ax^{2}y-3dxy^{2}-ay^{3}+c_{0}$$
$$v(x,y)=dx^{3}-3axy^{2}-3dxy^{2}+c_{1}$$
which is equivalent to
$$v(x,y)=3ax^{2}y-3dxy^{2}-ay^{3}+dx^{3}+c$$
where $c\in\R$. So, our adding $u$ and $v$ gives us $f$ which is
$$f(z)=(ax^{3}-3dx^{2}y-3axy^{2}+dy^{3})+i(3ax^{2}y-3dxy^{2}-ay^{3}+dx^{3})+c$$
\end{proof}
\end{document}