\documentclass[12pt]{article}
\usepackage{amssymb}
\usepackage{amscd}
\usepackage{amsxtra}
\usepackage{amsmath}
\usepackage{enumitem}
\newcommand{\N}{\mathbb{N}}
\newcommand{\C}{\mathbb{C}}
\newcommand{\R}{\mathbb{R}}
\newcommand{\Z}{\mathbb{Z}}
\newcommand{\Q}{\mathbb{Q}}
\newcommand{\ep}{\varepsilon}
\newcommand{\set}[1]{\left\{ #1\right\}}
\newenvironment{proof}{\noindent{\bf Proof.}}{\hfill $\square$\medskip}

\usepackage[utf8]{inputenc}


\title{Math 122A Homework 4}
\author{Rad Mallari}

\begin{document}
\maketitle

\section{Problem 1}
In each case find all the values of $z$ such that:
\begin{enumerate}[label=\textbf{(\alph*)}]
    \item $z=i^{i}$
    \item $z=(1-i)^{1+i}$
    \item $e^{\frac{1}{z}}=1+i\sqrt{3}$
\end{enumerate}

\begin{proof}
\begin{enumerate}[label=\textbf{(\alph*)}]
    \item By DeMoivre's formula we know that $i=e^{i(\frac{\pi}{2}+2\pi k)}$ where $k\in\Z$, since this occurs every $2\pi$. Therefore substituting this to the right side of our equation, we have that $z=(e^{i(\frac{\pi}{2}+2\pi k)})^{i}\Rightarrow z=e^{i^{2}(\frac{\pi}{2}+2\pi k)}=e^{-(\frac{\pi}{2}+2\pi k)}$.
    \item Similarly, using DeMoivre's formula, we know that $1-i=\sqrt{2}e^{-i(\frac{\pi}{4}+2\pi k)}$ where $k\in\Z$, which implies $z=(\sqrt{2}e^{-i(\frac{\pi}{4}+2\pi k)})^{(1+i)}=\sqrt{2}^{(1+i)}e^{-i(\frac{\pi}{4}+2\pi k)(1+i)}$.
    Distributing the exponents gives us $z=\sqrt{2}^{(1+i)}e^{(\frac{\pi}{4}+2\pi k)}\cdot e^{-i(\frac{\pi}{4}+2\pi k)}$. Again, by DeMoivre's formula we have that:
    $$z=\sqrt{2}^{(1+i)}\left(e^{(\frac{\pi}{4}+2\pi k)}\cos(-\frac{\pi}{4}-2\pi k)+i\sin(-\frac{\pi}{4}-2\pi k)\right)$$
    $$z=\sqrt{2}^{(1+i)}\cdot e^{(\frac{\pi}{4}+2\pi k)}\left(\frac{\sqrt{2}}{2}-i\frac{\sqrt{2}}{2}\right)$$
    $$z=(\sqrt{2})^{i}\cdot e^{(\frac{\pi}{4}+2\pi k)}\cdot(1-i)$$

    \item Finally, by the same formula we have that $1+i\sqrt{3}=2e^{i(\frac{\pi}{3}+2\pi k)}$ where $k\in\Z$, so our equation becomes $e^{\frac{1}{z}}=2e^{i(\frac{\pi}{3}+2\pi k)}$. Taking the natural log of both sides gives us $\frac{1}{z}=\ln(2e^{i(\frac{\pi}{3}+2\pi k)})\Rightarrow\frac{1}{z}=\ln(2)+i(\frac{\pi}{3}+2\pi k)$. Multiplying by $z$ and dividing by the right side to both sides yields, $$z=\frac{1}{\ln( 2)+i(\frac{\pi}{3}+2\pi k)}$$
    $$z=\frac{\ln (2)}{(\ln 2)^{2}+(\frac{\pi}{3}+2\pi k)^{2}}-i\left(\frac{\frac{\pi}{3}+2\pi k}{(\ln2)^{2}+(\frac{\pi}{3}+2\pi k)^{2}}\right)$$
\end{enumerate}
\end{proof}


\newpage
\section{Problem 2}
For any $z\in\C$ define:
$$\sin(z)=\frac{e^{iz}-e^{-iz}}{2i},\quad\cos(z)=\frac{e^{iz}+e^{-iz}}{2},\quad\sinh(z)=\frac{e^{z}-e^{-z}}{2},\quad\cosh(z)=\frac{e^{z}+e^{-z}}{2}$$
Prove:
\begin{enumerate}[label=\textbf{(\alph*)}]
    \item $\sin(-z)=-\sin(z),\qquad\cos(-z)=\cos(z)$
    \item $\sin^{2}(z)+\cos^{2}(z)=1,\qquad\cosh^{2}(z)-\sinh^{2}(z)=1$
    \item $\cos(z_{1}+z_{2})=\cos(z_{1})\cos(z_{2})-\sin(z_{1})\sin(z_{2})$
    \item $\cosh(z_{1}+z_{2})=\cosh(z_{1})\cosh(z_{2})+\sinh(z_{1})\sinh(z_{2})$
    \item $(\sin(z))'=\cos(z),\qquad(\cos(z))'=-\sin(z)$
    \item $(\sinh(z))'=\cosh(z)\qquad(\cosh(z))'=\sinh(z)$
\end{enumerate}

\begin{proof}
\begin{enumerate}[label=\textbf{(\alph*)}]
    \item By definition, we know that
    $$\sin(-z)=\frac{e^{-iz}-e^{iz}}{2i}.$$
    Meanwhile,
    $$-\sin(z)=-\left(\frac{e^{iz}-e^{-iz}}{2i}\right)=\frac{e^{-iz}-e^{iz}}{2i},$$
    which is exactly $\sin(-z)$ therefore $\sin(-z)=-\sin(z)$.

    Now,
    $$\cos(-z)=\frac{e^{-iz}+e^{iz}}{2}=\frac{e^{iz}+e^{-iz}}{2},$$
    by definition which is exactly $\cos(z)$.
    \item By definition we have that:
    $$\sin^{2}(z)=\frac{-e^{2iz}-e^{-2iz}}{4}+\frac{1}{2}$$
    and
    $$\cos^{2}(z)=\frac{e^{2iz}+e^{-2iz}}{4}+\frac{1}{2}$$
    Therefore, adding $\sin^{2}(z)+\cos^{2}(z)$ yields
    $$\sin^{2}(z)+\cos^{2}(z)=\frac{-e^{2iz}-e^{-2iz}}{4}+\frac{1}{2}+\frac{e^{2iz}+e^{-2iz}}{4}+\frac{1}{2}$$
    $$\sin^{2}(z)+\cos^{2}(z)=1$$
    Similarly, by definition, we know
    $$-\sinh^{2}(z)=-\left(\frac{e^{2z}+e^{-2z}}{4}-\frac{1}{2}\right)=-\frac{-e^{2z}-e^{-2z}}{4}+\frac{1}{2}$$
    and
    $$\cosh^{2}(z)=\frac{e^{2z}+e^{-2z}}{4}+\frac{1}{2}$$
    So adding them together gives us
    $$\cosh^{2}(z)+-\sinh^{2}(z)=\frac{e^{2z}+e^{-2z}}{4}+\frac{1}{2}-\frac{-e^{2z}-e^{-2z}}{4}+\frac{1}{2}$$
    $$\cosh^{2}(z)+-\sinh^{2}(z)=1$$
    \item We know that, by definition:
    $$\cos(z_{1})\cos(z_{2})=\frac{e^{i(z_{1}+z_{2})}+e^{i(z_{1}-z_{2})}+e^{i(z_{2}-z_{1})}+e^{i(-z_{1}-z_{2})}}{4}$$
    and
    $$-\sin(z_{1})\sin(z_{2})=\frac{e^{i(z_{1}+z_{2})}-e^{i(z_{1}-z_{2})}-e^{i(z_{2}-z_{1})}+e^{i(-z_{1}-z_{2})}}{4}$$
    Therefore,
    $$\cos(z_{1})\cos(z_{2})-\sin(z_{1})\sin(z_{2})=\frac{2e^{i(z_{1}+z_{2})}+2e^{i(-z_{1}-z_{2})}}{4}$$
    $$\cos(z_{1})\cos(z_{2})-\sin(z_{1})\sin(z_{2})=\frac{e^{i(z_{1}+z_{2})}+e^{i(-z_{1}-z_{2})}}{2}$$
    Now
    $$\cos(z_{1}+z_{2})=\frac{e^{i(z_{1}+z_{2})}+e^{i(-z_{1}-z_{2})}}{2}$$
    which is exactly our result from $\cos(z_{1})\cos(z_{2})-\sin(z_{1})\sin(z_{2})$.
    \item Taking the derivative of $\sin(z)$
    $$(\sin(z))'=\frac{d}{dz}\left(\frac{e^{iz}}{2i}\right)-\frac{d}{dz}\left(\frac{e^{-iz}}{2i}\right)$$
    $$(\sin(z))'=\frac{ie^{iz}}{2i}-\frac{-ie^{-iz}}{2i}$$
    $$(\sin(z))'=\frac{e^{iz}+e^{-iz}}{2}$$
    Which is exactly the definition of $\cos(z)$
    And for $\cos(z)$
    $$(\cos(z))'=\frac{d}{dz}\left(\frac{e^{iz}}{2}\right)+\frac{d}{dz}\left(\frac{e^{-iz}}{2}\right)$$
    $$(\cos(z))'=\frac{ie^{iz}}{2}+\frac{-ie^{-iz}}{2}$$
    Multiplying by $\frac{i}{i}$, we get
    $$(\cos(z))'=\frac{-e^{iz}+e^{-iz}}{2i}$$
    Which is exactly the definition of $-\sin(z)$
    \item Again, taking the derivative of $\sinh(z)$
    $$(\sinh(z))'=\frac{d}{dz}\left(\frac{e^{z}}{2}\right)-\frac{d}{dz}\left(\frac{e^{-z}}{2}\right)$$
    $$(\sinh(z))'=\frac{e^{z}}{2}-\frac{-e^{-z}}{2}=\frac{e^{z}+e^{-z}}{2}$$
    which is the definition of $\cosh(z)$.
    Finally, taking the derivative of $\cosh(z)$ gives us
    $$(\cosh(z))'=\frac{d}{dz}\left(\frac{e^{z}}{2}\right)+\frac{d}{dz}\left(\frac{e^{-z}}{2}\right)$$
    $$(\cosh(z))'=\frac{e^{z}}{2}+\frac{-e^{-z}}{2}=\frac{e^{z}-e^{-z}}{2}$$
    which is the definition of $\sinh(z)$.
    \item Finally, taking the derivative of $\sinh(z)$, we have that
    $$(\sinh(z))'=\frac{e^{z}}{2}+\frac{e^{-z}}{2}=\frac{e^{z}+e^{-z}}{2}=\cosh(z)$$
    Furthermore, taking the derivative of $\cosh(z)$ yields:
    $$(\cosh(z))'=\frac{e^{z}}{2}+\frac{-e^{-z}}{2}=\frac{e^{z}-e^{-z}}{2}=\sinh(z)$$
\end{enumerate}
\end{proof}


\newpage
\section{Problem 3}
Evaluate the following integrals ($k\in\Z$):
\begin{enumerate}[label=\textbf{(\alph*)}]
    \item $\int_{1}^{2}(\frac{1}{t}+i)^{2}dt$
    \item $\int_{0}^{\frac{\pi}{3}}e^{it}dt$
    \item $\int_{0}^{2\pi}e^{ikt}dt$
\end{enumerate}

\begin{proof}
\begin{enumerate}[label=\textbf{(\alph*)}]
    \item
    $$\int_{1}^{2}\left(\frac{1}{t}+i\right)^{2}dt=\int_{1}^{2}\left(\frac{1}{t^{2}}+2i\frac{1}{t}-1\right)dt$$
    $$=-(-\frac{1}{2})+2i(\ln(2)-\ln(1))-1$$
    $$=-\frac{1}{2}+2i\ln(2)$$
    \item
    $$\int_{0}^{\frac{\pi}{3}}e^{it}dt=\frac{1}{i}\left(e^{i\frac{\pi}{3}}-1\right)$$
    $$=\frac{1}{i}(\cos\left(\frac{\pi}{3}\right)+i\sin\left(\frac{\pi}{3}\right)-1)$$
    $$=-\frac{1}{2i}\left(\frac{i}{i}\right)+\frac{\sqrt{3}}{2}$$
    $$=\frac{i+\sqrt{3}}{2}$$
    \item
    $$\int_{0}^{2\pi}e^{ikt}dt=\frac{i}{ik}(e^{i2\pi k}-1)$$
    $$=\frac{1}{ik}(1-1)$$
    $$=0$$
\end{enumerate}
\end{proof}


\section{Problem 4}
In each case write the equation of the curve representing:
\begin{enumerate}[label=\textbf{(\alph*)}]
    \item The segment joining $1$ and $i$
    \item The circumference of center $1-i$ and radius $2$, in the counter clockwise direction
    \item The triangle with vertices $1, i, -2$.
\end{enumerate}

\begin{proof}
\begin{enumerate}[label=\textbf{(\alph*)}]
    \item We know that $z_{0}=1$ and $z_{1}=1+i$, then parameterizing our $z$ we have
    $$\gamma:[0,1]\to\C$$
    $$t:[0,1]\to ti+(1-t)$$
    \item Similarly, we have a circle of radius $2$, centered at $1-i$, so our equation is
    $$\left|z-(1-i)\right|=2$$
    Parameterizing this we have
    $$z(t)=(1-i)+2e^{it}\qquad \text{for }0\leq t\leq 2\pi$$
    \item For this we split the parameterization to three pieces $\gamma=\gamma_{1}+\gamma_{2}+\gamma_{3}$ where $\gamma_{1}:[0,1]\to\C$, $\gamma_{2}:[0,1]\to\C$, $\gamma_{3}:[0,1]\to\C$ and $\gamma_{1}(t)=ti+(1-t)$, $\gamma_{2}(t)=(1-t)i-2t$, $\gamma_{3}(t)=3t-2$.
\end{enumerate}
\end{proof}


\newpage
\section{Problem 5}
Evaluate the following integrals:
\begin{enumerate}[label=\textbf{(\alph*)}]
    \item $\int_{\gamma}xdx$, $\gamma$ the boundary on the unit square.
    \item $\int_{\gamma}e^{z}dz$, $\gamma$ the portion of the unit circle joining $1$ and $i$ in the counter clockwise direction.
    \item $\int_{\gamma}xdy$, $\gamma$ is the boundary of a bounded region $A\subset\R^{2}$ (without holes) in the counter clockwise direction.\\\textbf{HINT:} Use Green's Theorem.
\end{enumerate}

\begin{proof}
\begin{enumerate}[label=\textbf{(\alph*)}]
    \item Similar to \textbf{Problem 5(c)}, we have four parts we parameterize our curve which is $\gamma=\gamma_{1}+\gamma_{2}+\gamma_{3}+\gamma_{4}$ where each $\gamma$ are given by:
    \begin{flalign}
        &\gamma_{1}:[0,1]\to\C\quad\Rightarrow\quad\gamma_{1}(t)=(1-t)+(1+i)t=1+it \nonumber \\
        &\gamma_{2}:[0,1]\to\C\quad\Rightarrow\quad\gamma_{2}(t)=1-t+i \nonumber \\
        &\gamma_{3}:[0,1]\to\C\quad\Rightarrow\quad\gamma_{3}(t)=(1-t)i=i-ti \nonumber\\
        &\gamma_{4}:[0,1]\to\C\quad\Rightarrow\quad\gamma_{4}(t)=t \nonumber
    \end{flalign}
    Now parameterize the integral $\int_{\gamma} xdz=\int_{0}^{1} f(\gamma_{n}(t))\dot{\gamma_{n}}dt$
    \begin{flalign}
        &\int_{0}^{1}f(\gamma_{1}(t))\dot{\gamma_{1}}dt=\int_{0}^{1}idt=i \nonumber \\
        &\int_{0}^{1}f(\gamma_{2}(t))\dot{\gamma_{2}}dt=\int_{0}^{1}(1-t)(-1)dt=-\frac{1}{2} \nonumber \\
        &\int_{0}^{1}f(\gamma_{3}(t))\dot{\gamma_{3}}dt=\int_{0}^{1}(0)(-i)dt=0 \nonumber \\
        &\int_{0}^{1}f(\gamma_{4}(t))\dot{\gamma_{4}}dt=\int_{0}^{1}tdt=\frac{1}{2} \nonumber
    \end{flalign}
    Therefore, $\int_{\gamma} xdz=i$.
    \item By the theorem in section 48 and lecture 7, we know that $\int e^{z}dz$ can be given by $F(b)-F(a)$, therefore $\int_{\gamma}e^{z}dz=e^{i}-e$
    \item Green's Theorem states that:
    $$\oint_{C} P_{(x,y)}dx+Q_{(x,y)}dy\equiv\iint\limits_{\Omega}(\partial_{x}Q-\partial_{y}P)dxdy$$
    Where in our case, $\Omega=A$, $P=0$, and $Q=x$ giving us
    $$\oint xdy=\iint\limits_{A}(\partial_{x}x)dA=A$$
\end{enumerate}
\end{proof}
\end{document}